% !TeX document-id = {04d65a32-516d-4b85-9aaa-0206108e4b3a}
%!TEX program = xelatex 
% !BIB program = biber
\PassOptionsToPackage{table}{xcolor}
\documentclass[cn,10pt,citestyle=gb7714-2015, bibstyle=gb7714-2015]{elegantbook}
% citestyle=gb7714-2015 表示使用上标
% https://www.ctan.org/pkg/biblatex-gb7714-2015?lang=en


% 配色
% third: proposition
% 0, 160, 152   (green)
% 244, 105, 102  (cyan)
% 0, 174, 247    (blue)

% second:
% 175, 153, 8  (cyan)
% 230, 90, 7   (green)
% 255, 134, 24 (blue)



\title{运动生物力学}
\subtitle{开源湖工商经典之作}

\institute{OpenHUTB}
\date{\today}
\version{2.0}

\extrainfo{认识你自己。——古希腊德尔斐神庙墙上镌刻的箴言}

\setcounter{tocdepth}{3}

%\logo{logo-blue.png}
\cover{cover.jpg}

% 本文档命令
\usepackage{array}
% \usepackage{ctex}%加载ctex宏包,中文支持
%\usepackage{xr}

\newcommand{\ccr}[1]{\makecell{{\color{#1}\rule{1cm}{1cm}}}}

\definecolor{customcolor}{RGB}{32,178,170}
\colorlet{coverlinecolor}{customcolor}

\bibliography{reference}  % 关联参考文献文件 reference.bib

\begin{document}

\maketitle
\frontmatter

\chapter*{前言}

\markboth{Introduction}{前言}



\vskip 1.5cm

像诺亚方舟一样,开始一个巨大而愚蠢的项目……人们对你的看法完全无关紧要。


\vskip 0.5cm


%\vskip 1.5cm

\begin{flushright}
—— 鲁米\\
\end{flushright}


生物力学一直是我生活中积极而持久的力量。
小时候,在德尔普家,运动占据了我的生活。打棒球和长曲棍球让我充满活力,也喜欢跑步和滑雪。
高中毕业前,我只读过 2 本书:
一本是高山滑雪技术手册,另一本是跳远教练指南。
我有阅读障碍,只有这些简陋的生物力学手册,才值得我费力费力地去读,即使有些尴尬。


我在大学学习生物力学,学习如何在一次滑雪事故中髋部受伤后恢复行走能力。
我学习了所有关于生物力学的知识,几年后,当我能够行走自如,在斯坦福大学校园里一瘸一拐地走动时,我进入了研究生院。
我在研究生院学习了设计、机器人技术、神经科学和生物力学。
我觉得自己很适合帮助行动障碍人士,因为我对他们的困境感同身受,并且拥有深厚的工程设计和计算机科学背景,这在当时的生物医学研究中并不常见。
我有幸与费利克斯$\cdot$扎亚克一起学习生物力学,他对我的思考产生了很大的影响。



从斯坦福大学毕业后,我到西北大学和芝加哥康复研究所担任助理教授。
作为一名新教授,我希望分享我对生物力学的热爱,并教授一些能够帮助学生理解和分析肌肉和运动的原理。
这促成了我开设“运动生物力学”课程,这门课程我在西北大学和斯坦福大学已经讲授了大约30次。
本书涵盖了我在这门课程中教授的大部分内容。


通过阅读本书,你将学习生物力学的一些最重要的原理。
生物力学是一个运用物理学解释生物体运作方式的领域,从而帮助我们理解和珍惜生命。
生物力学是许多学科的核心,包括生物工程、机械工程、物理治疗、人体工程学、运动机能学和生物学。
我希望这些领域的学生能够从本书中受益。
我也为对神经科学、机器人技术、计算机科学和体育感兴趣的读者撰写本书。
除了学习生物力学原理外,你还将获得分析和模拟人体运动的技能;
这些技能将赋予你力量,让你为世界带来积极的改变。


许多刚开始学习生物力学的人分为 2 类:
一类是工程师,他们对生物学的语言感到困惑;
另一类是生物学家,他们对工程学的语言感到困惑。
这两种语言都很重要。
虽然它们经常被分开教授,但大自然并不在意人类在学科之间设置的界限。
本书的目的之一就是帮助你在精通这两门学科方面有一个良好的开端。
因此,我并不要求你事先具备任何生物学知识。


我确实需要假设读者对物理学有所了解,否则这本书会比现在长得多。
牛顿定律描述了从行星到人体的一切运动。
我们需要这些定律以及其他力学原理来分析人体运动,因此我假设读者已经学习过物理或工程学课程,具备力学方面的基础知识。
由于数学是力学的语言,我还假设读者掌握了微积分的基本概念,包括导数、积分和微分方程。
毕竟,力学的起点——牛顿第二定律,本身就是一个微分方程。


我的目标是写一本通俗易懂的书,用引人入胜、发人深省的例子来阐述生物力学的关键概念。
我写这本书并非为了综述文献。
关于生物力学的文章数以千计,引人入胜,也有很多优秀的综述文章;
我只引用了其中的一小部分。
我的目标是向生物力学的新手介绍生物力学,并帮助他们探索生物力学文献。
除了涵盖既定的原理外,我还重点介绍了一些将塑造该领域未来的重大理念,并阐述了如何践行这些理念,从而对科学、工程、艺术以及他人的生活产生积极影响。
如果您已经是生物力学领域的专家,我衷心希望您喜欢这本书,并发现它对您的教学大有裨益。


我希望为您提供解决实际问题的工具。
虽然本书没有印刷作业题,但我和我的同事开发了一个网站,其中包含许多有趣且具有挑战性的题目。
解决这些问题将大大扩展您的知识面。
您可以在 \href{biomech.stanford.edu}{biomech.stanford.edu} 在线找到这些问题,该网站还提供讲座幻灯片、教学大纲和其他欢迎您使用的资料。
您也可以在此网站上提供反馈。
请告诉我您喜欢哪些内容以及我可以如何改进本书。
本书包含大量数据,我和我的同事在 \href{simtk.org}{simtk.org} 上提供了其中的大部分数据。
我们建立这个网站是为了分享生物力学模型、软件和数据,我鼓励您使用并为之做出贡献。


这本书包含大量罕见汇集的信息,但内容并不全面。
我很少详细阐述神经系统在协调和控制运动中的作用,也没有讨论肌肉如何随着运动、废用或年龄而变化。
此外,我对骨骼、软骨和韧带的生物力学也进行了有限的讨论。
这些引人入胜的主题在其他书籍中都有详尽的阐述。


与汤姆$\cdot$内田和我的兄弟大卫$\cdot$德尔普一起撰写和绘制这本书是我职业生涯中最愉快、最有成就的合作之一。
大部分工作都是在我们家餐桌上的“聚会”中完成的。
我们的目标之一是用同一个声音写作。
汤姆和我合作得非常密切,以至于我们都无法告诉你是谁起草了这本书的句子。
这些故事来自我的经历,由于我曾多次教授运动生物力学课程,我们决定用我的风格来写;
然而,每一页都留下了汤姆的指纹。
如果您欣赏书中对力学概念表达的精准性,您可以为此感谢汤姆。
如果您和大多数人一样更喜欢书中的图片而不是文字,您可以感谢我的兄弟。
他是一位出色的艺术家,帮助我们创作了这本书,希望您会发现它是一本精美的书。


数学家兼才华横溢的作家达纳$\cdot$麦肯齐对整部手稿提供了详尽的评论和建议,极大地改进了本书。
读书不应该成为一件苦差事,而应该带给读者与作者同等的快乐。
达纳努力从那些隐藏着快乐的地方唤起这份快乐。
我非常感谢他这样做,我想你也会如此。


世界级生物力学家 Silvia Blemker 提供了第 4 至 6 章的早期草稿,其中包括一些关于图表和示例的精彩构思,她值得特别感谢。
我们还要感谢麻省理工学院出版社的 Molly Seamans 和 Matthew Abbate。
Molly 为本书的设计提供了创意,并负责了每一页的排版。
Matthew 则对文本提出了宝贵的反馈,并逐字逐句地审校。



许多例子都出自我实验室的研究成果,我的合作者需要得到认可,为此,我在书中列举了一些合作者的名字,并在我们合著的论文中引用了其他合作者的名字。
美国国立卫生研究院 (NIH) 为这项工作提供了支持,我非常感谢他们的支持。


其他朋友、同事、学生和导师也做出了宝贵的贡献,他们审阅了本书的初稿,与我合作撰写了一些论文,并从中汲取了灵感,或者向我解释了一些关键概念。
我要感谢以下几页图片中出现的各位,感谢他们的真知灼见和合作。
生物力学是一项团队运动,我很荣幸能与你们并肩作战。



我非常感谢在生物力学领域结识的众多朋友,他们一直是我生活中的一股积极力量。




\tableofcontents

\mainmatter



\hypersetup{bookmarksopen=false}

\part{运动}
\markboth{运动}{运动}

\chapter{第一步} \label{chap:chap1}

当时已是凌晨一点钟;雨点凄厉地拍打着窗玻璃,我的蜡烛快要燃尽了。
这时,借着半熄灭的微光,我看到那怪物睁着一只暗黄色的眼睛;它呼吸急促,四肢抽搐着。

\begin{flushright}
	——玛丽$\cdot$雪莱,《弗兰肯斯坦》(1818)\\
\end{flushright}


\begin{figure}[!htb]
	\centering
	\includegraphics[width=0.5\linewidth]{chap1/1_0}
	% 加星号(*)表示不加编号
	\caption*{ \label{fig:1_0}}
\end{figure}

当玛丽$\cdot$雪莱写下那本让她名声大噪的哥特式小说时,全世界仍在为电的发现而兴奋不已,而路易吉$\cdot$加尔瓦尼惊人发现:电刺激竟能使死青蛙的肌肉抽搐。
雪莱几乎可以想象,电流能使一个完整的生命体复活。


如今,距离伽伐尼的演示已过去两个世纪,肌肉电刺激技术对生活产生了深远的积极影响。
每个机场、学校和医院都安装了电除颤器,它们已经让成千上万的人的心脏重新跳动,而如果没有对心肌进行强力电击,这些人可能会死亡。
一项名为功能性电刺激的更先进技术已被开发出来,它可以使各种瘫痪患者的肌肉恢复活力,从而改善他们的生活质量。


在功能性电刺激中,电极要么贴在皮肤上,要么植入瘫痪肌肉中,靠近支配这些肌肉的神经。
电脉冲通过电极传输到神经,进而引起相关肌肉收缩并产生力量。
肌肉电刺激使瘫痪患者在躯干和腿部肌肉得到适当激活和协调的情况下能够站立和行走。
2016 年,几名脊髓损伤导致胸部以下瘫痪的患者参加了一项名为“Cybathlon”的全新国际自行车比赛。
获胜者在不到 3 分钟的时间内骑行了 750 米。


功能性电刺激远不止向神经和肌肉输送电流那么简单。
对于健全人来说,神经系统协调着许多肌肉,使我们能够行走、跑步或骑自行车,就像指挥家协调管弦乐队的乐手一样。
当 Cybathlon 比赛的冠军在赛道上骑行时,能够传送多达 24 个通道的精确定时刺激,以协调他原本瘫痪的肌肉产生的力量。
即使达到了这种复杂程度,结果也并非完美。
关于我们的肌肉如何协同工作以创作运动之乐,我们仍有许多需要学习的地方。


在本书中,我们将探索这首宏伟的交响曲。
我们将从力学的视角审视人类和动物的运动,以理解运动的生物力学。
我们将运用简单的概念模型来解答诸如:为何行走和跑步是高效的运动方式、为何宇航员在月球上采取跳跃式步态,以及跑道和跑鞋如何提升运动表现并减少损伤等问题。
我们将深入研究肌肉的结构,直至其微观的动力产生机制,并精确观察电刺激如何促使肌肉收缩。
我们将描述用于生成运动模拟的复杂计算工具,从而使我们能够估算产生这种运动的肌肉力量。
这些模拟向我们展示了我们在行走、跑步或骑自行车时如何协调肌肉。
事实上,肌肉驱动的模拟对于 Cybathlon 金牌团队来说是一个宝贵的工具。


本书始终强调既定理论,为理解运动生物力学奠定基础,并涵盖计算机模拟、移动运动监测和可穿戴机器人等领域基于这些基础的创新。
许多近期的进展都由科幻小说所预见,并为我们展现未来新技术的雏形。
其中一些愿景或许如同唤醒弗兰肯斯坦的怪物般奇幻,而另一些则近在眼前。
无论如何,我们都在探索这个激动人心领域的潜力。






\section{我们为什么研究运动}

运动令人着迷,是生命的基础。
我们的身体功能多样,既能展现力量,又能展现灵巧。
运动对于维持身心健康至关重要。
规律的体育锻炼有助于预防心脏病、癌症、骨质疏松症、肥胖症、糖尿病、抑郁症、焦虑症和其他严重疾病,然而,全球只有不到一半的人口能够充分活动以维持身心健康。
体育锻炼是一剂强效且廉价的良药,即使少量运动也能带来显著的健康益处。
在研究运动生物力学的过程中,我们致力于理解运动产生的生物结构和过程,并将这些知识应用于提高灵活性、体育锻炼能力和健康水平。


运动生物力学领域历史悠久。
如同人类的许多追求一样,推动其进步的动力源于改善生活的渴望,以及对环境和自身与生俱来的好奇心。
亚里士多德在公元前 350 年左右撰写了第一本探讨动物运动一般原理的著作,书名恰如其分地命名为《动物运动论》。
他和其他古希腊人认为,当肌肉被“气”(pneuma)(流经我们神经的“生命之气”)充气时,肌肉就会收缩。


此后,无数学者推动了该领域的发展,本书将介绍其中一些学者。
生物力学的先驱包括列奥纳多$\cdot$达$\cdot$芬奇(1452-1519),他绘制了数百幅详细的解剖图,并研究了肌肉骨骼系统的力学功能(图~\ref{fig:1_1})。
乔瓦尼$\cdot$博雷利(1608-1679)是第一个运用力学定律将肌肉施加的力与其在关节周围产生的力矩联系起来的人(图~\ref{fig:1_2})。
然而,博雷利仍然坚持经典观点,认为肌肉是通过气动、膨胀过程收缩的。
这一理论遭到了佛罗伦萨宫廷中博雷利的对手尼古拉斯$\cdot$斯坦诺(Nicolas Steno,1638-1686)的反驳,他指出肌肉在收缩时体积保持不变,我们将在第 4 章中研究这个悖论。
后来,路易吉$\cdot$加尔瓦尼(Luigi Galvani,1737-1798)发现了电信号能够引起肌肉收缩这一此前未被怀疑的能力,从根本上奠定了我们现代电生理学领域的基础。
最后,我不能不提一下埃德沃德$\cdot$迈布里奇(Eadweard Muybridge,1830-1904),他在距离我家约一公里(现在的斯坦福大学校园)的地方进行了早期的人类和动物运动摄影研究。迈布里奇的照片即使在今天看起来也令人着迷,它们是电影和生物力学史上的里程碑(图~\ref{fig:1_3})。
事实上,电影制作技术和生物力学科学仍在齐头并进。


\begin{figure}[!htb]
	\centering
	\includegraphics[width=0.75\linewidth]{chap1/1_1}
	\caption{列奥纳多$\cdot$达$\cdot$芬奇笔记本中的一页,展示了他的肌肉力线概念。
		图片由皇家收藏信托基金会提供。 \label{fig:1_1}}
\end{figure}


\begin{figure}[!htb]
	\centering
	\includegraphics[width=0.75\linewidth]{chap1/1_2}
	\caption{乔瓦尼$\cdot$博雷利(Giovanni Borelli)对人体肌肉和关节的静态分析,他常被誉为生物力学之父。
		图片来自《动物运动》(De motu animalium)。 \label{fig:1_2}}
\end{figure}


\begin{figure}[!htb]
	\centering
	\includegraphics[width=1.0\linewidth]{chap1/1_3}
	\caption{动作捕捉先驱埃德沃德$\cdot$迈布里奇的一系列照片。
		图片由斯坦福大学提供。  \label{fig:1_3}}
\end{figure}


如今,生物力学是一个快速发展的多学科领域,汇聚了众多科学和工程领域的专家学者的通力合作。
生物学家利用生物力学的洞见来理解动物形态与功能之间的关系:
例如,蜥蜴如何跳跃并抓住墙壁(图~\ref{fig:1_4}),或者霸王龙是否能够奔跑(第~\ref{chap:chap3}~章)。
神经科学家研究大脑如何在运动过程中协调肌肉,以及这些神经回路在受伤和患病的情况下如何受到干扰。
外科医生可以使用生物力学模型来确定脑瘫患者是否能从肌腱延长手术中受益。
机器人专家正在发明日益精密的假肢(图~\ref{fig:1_5})和能够在危险环境中执行复杂任务的双足机器人。
运动科学家分析运动动作,例如“背越式跳高”(第~\ref{chap:chap9}~章),以了解如何提高运动表现并预防伤病。
计算机科学家和生物力学工程师开发新的算法和软件工具来模拟运动并从这些模拟中获得见解。


\begin{figure}[!htb]
	\centering
	\includegraphics[width=1.0\linewidth]{chap1/1_4}
	\caption{红头鬣蜥在飞行过程中用尾巴控制身体方向。
		这只蜥蜴准备抓住右侧的垂直墙壁,它顺时针旋转尾巴,通过角动量守恒使身体保持垂直。
		图片由罗伯特$\cdot$富尔提供。 \label{fig:1_4}}
\end{figure}


\begin{figure}[!htb]
	\centering
	\includegraphics[width=0.5\linewidth]{chap1/1_5}
	\caption{功能强大的假肢如今触手可及。
		图片由约翰$\cdot$霍普金斯大学提供。 \label{fig:1_5}}
\end{figure}


生物力学甚至在科学领域之外也发挥着作用。电影制作人运用生物力学和动作捕捉技术,为游戏和电影创作计算机生成的图像,风格各异,从奇幻到逼真(图~\ref{fig:1_6})。
其成果将美感与科学的准确性完美结合,这在一两代人之前是难以想象的。


\begin{figure}[!htb]
	\centering
	\includegraphics[width=1.0\linewidth]{chap1/1_6}
	\caption{图片来自电影《阿凡达》和行为艺术家詹$\cdot$斯塔福德。
		她的动作被动作捕捉技术记录下来,并制作成计算机生成的图像。
		图片由二十世纪福克斯公司提供。 \label{fig:1_6}}
\end{figure}


总的来说,这些努力极大地改善了我们的生活。
现在,我们可以根据保持健康所需的日常体力活动水平获得建议,并且我们可以使用有助于我们实现健身目标的工具。
装配线、办公家具和许多消费品都符合人体工程学设计,以提高舒适度并防止受伤。
一些产品和程序已被设计用于置换髋关节和膝关节,减轻疼痛并恢复数百万骨关节炎患者的功能。
动力外骨骼正在彻底改变中风后康复,并可以使瘫痪患者恢复运动能力。
帕金森病等运动障碍已通过深部脑刺激得到成功治疗,深部脑刺激是指植入电极向大脑特定区域传递电脉冲。
硬膜外刺激是一种激活脊髓神经回路的技术,最近已显示出帮助脊髓损伤患者恢复自主运动的潜力。
运动员正受益于旨在降低受伤风险并提高运动表现的设备和训练计划(图~\ref{fig:1_7})。


\begin{figure}[!htb]
	\centering
	\includegraphics[width=0.6\linewidth]{chap1/1_7}
	\caption{生物力学研究有助于打造能够最大限度提升运动表现的运动器材。
		这款冰鞋的靴子和冰刀之间的铰链设计延长了冰刀与冰面的接触时间,从而增加了推进力的持续时间,从而提高了速度。
		图片由 McSmit 提供。 \label{fig:1_7}}
\end{figure}



\section{半机械人奥运会}


或许,更深入地观察一个引人注目的例子——“Cybathlon”(人机合体竞技),就能更好地理解生物力学的影响。
2016年,苏黎世联邦理工学院(ETH Zurich)教授罗伯特$\cdot$里纳(Robert Riener)组织了首届“Cybathlon”,这是一项将科学与体育进行创新性融合的运动。
与残奥会不同,这项运动的比赛项目侧重于日常任务。
例如,佩戴假肢的运动员比赛搬运物品、切面包、开罐子和晾衣服。
佩戴假肢的运动员比赛爬坡和上楼梯,同时保持杯子在碟子上保持平衡。
为了强调这项比赛关乎人类与科技的结合,参赛者被称为“飞行员”,而不是“运动员”。


在这场非传统的比赛中,最传统的体育项目是腿部瘫痪者自行车赛。
金牌被凯斯西储大学在罗纳德$\cdot$特里奥洛(Ronald Triolo)的科研领导下夺得。
该队的车手是马克$ \cdot $穆恩(Mark Muhn)(图~\ref{fig:1_8}),他在2008年的一次滑雪事故中脊髓受伤,胸部以下瘫痪。


\begin{figure}[!htb]
	\centering
	\includegraphics[width=0.6\linewidth]{chap1/1_8}
	\caption{马克$\cdot$穆恩参加 Cybathlon 比赛。
		图片由 Paul 和 Gabrielle Marasco 提供。 \label{fig:1_8}}
\end{figure}


凯斯西储大学的团队在功能性电刺激领域研究了四十年,并开发出了首批植入式神经肌肉刺激器。
特里奥洛相信,他们在植入电极方面的经验将成为制胜优势,因为其他团队正在使用通过皮肤传输电信号的电极。
植入电极可以更精准地刺激被激活的肌肉,从而产生更强烈的肌肉收缩。


该团队通过多种方式最大限度地提高了获胜的几率(McDaniel 等人,2017)。
他们购买了一辆卧式自行车,并将其拆解,去掉了所有不必要的重量。
他们让飞行员(最初有五名,其中两名被选中前往苏黎世)参加了强化训练计划。
我们最感兴趣的是他们使用的生物力学模型(图~\ref{fig:1_9})。


\begin{figure}[!htb]
	\centering
	\includegraphics[width=1.0\linewidth]{chap1/1_9}
	\caption{用于调整肌肉兴奋模式以产生循环的肌肉驱动生物力学模型的示例。 \label{fig:1_9}}
\end{figure}


特里奥洛表示,这些模型与我们在本书后面描述的类似,在两个方面发挥了作用:
首先,它消除了“死胡同”(行不通的想法);
其次,它优化了提供给飞行员的电刺激模式。
每块腿部肌肉的每一次收缩都由一个外部控制装置(一个绑在飞行员腰间的盒子)控制。
为了安全起见,飞行员可以控制盒子的开关。


为了获得刺激模式,研究小组从文献中获取了自行车运动的生物力学模型,并根据每位骑手的特点进行了定制。
定制非常重要,因为瘫痪后肌肉的特性会发生变化,而可以通过刺激激活的肌肉数量很少,所以适用于健全骑手的激活模式可能不适用于脊髓损伤患者的刺激驱动踩踏。
Musa Audu 带头为每位骑手建模和定制刺激模式,并采用第~\ref{chap:chap10}~章中描述的技术来估计每块肌肉激活的时间和强度。
值得注意的是,刺激模式在比赛期间从未改变。
当骑手的肌肉疲劳时,他的腿会保持以相同的速率抽动,但它们无法用那么大的力气推动,所以骑手必须换挡才能保持自行车前进。


事实上,飞行员之所以会很快感到疲劳,是因为功能性电刺激并不像大脑发出的自然信号那样募集肌肉纤维。
我们将在第~\ref{chap:chap4}~章中回顾这一现象,但现在只需说明,大多数自然发生的运动都是通过首先募集“慢肌”纤维(相对较小且抗疲劳),然后是“快肌”纤维(产生较大力量但很快疲劳)产生的。
然而,电刺激以相反的顺序募集肌肉纤维。
由于这种“反向募集”,飞行员在开始训练时几乎无法让摩托车持续行驶超过一分钟。



但意想不到的事情发生了——相比于单调乏味的固定自行车,飞行员们更喜爱比赛用自行车带来的户外锻炼。
而且运动训练可以增强瘫痪的肌肉。在飞行员们积极进取的激励下,经过5个月的训练,他们能够坚持骑车参加3分钟的比赛。
特里奥洛表示,他希望在未来几年找到一种技术解决方案来解决招募逆转问题,但在2016年,只有一个解决方案:“让我们的飞行员尽情锻炼”。
而且,这个方案真的奏效了!
在苏黎世,穆恩以2分58秒的成绩完成了750米的比赛,并夺得了金牌。


Cybathlon 的经历改变了 Muhn 的人生,或许更令人惊讶的是,它改变了 Triolo 的研究。
Triolo 说,以前他们的康复方法是任务导向的。
他们专注于让志愿者站立、行走或进行日常生活活动,比如穿衣和做饭。
但当他们的参与者在户外骑自行车,并开始在锻炼的同时享受乐趣时,一切都改变了。
他们锻炼得更多,这对他们康复的各个方面都产生了巨大的回报,包括提升了他们的自尊心。
当一名骑手在公共道路上骑自行车时,另一个骑手追上他并说:“轮子真漂亮。”
这几乎是陌生人第一次将他视为一个拥有酷炫自行车的人,而不是一个残疾人。



\section{研究运动的工具}

我写这本书的目标之一是让你熟悉我的团队和其他人员开发的肌肉驱动生物力学模型。重要的是要意识到这些模型是基于实验数据的。让我们来看看我们收集的数据类型。



分析运动的一种常用技术是在研究实验室和诊所录制个体的视频。
许多此类记录都是通过红外摄像机获得的,这些摄像机可以追踪贴在皮肤上的标记物,类似于电影制作中使用的技术(图~\ref{fig:1_10})。
最近,不需要标记物的运动捕捉技术变得越来越流行。
基于视频的系统已经广泛应用,但这些系统的普及程度已被惯性测量单元 (IMU) 所取代,IMU 现已集成到智能手机、可穿戴活动监测器和服装中。
IMU 能够在自然环境中长时间收集运动数据(速度和加速度),这对于监测病情进展和制定治疗方案非常有价值。
低成本活动监测器中 IMU 的普及也使得对全球数百万个体进行大规模研究成为可能(图~\ref{fig:1_11})。


\begin{figure}[!htb]
	\centering
	\includegraphics[width=1.0\linewidth]{chap1/1_10}
	\caption{前手翻(时长 2.9 秒)期间,贴在皮肤上的标记物轨迹。
		受试者最初站立(最左侧),然后向前跳跃,双手撑地翻身(中间),双脚落地,然后跳了一跳恢复平衡(最右侧)。
		球体间距越大,表示标记物移动速度越快。
		数据来自 ACCAD (2018)。 \label{fig:1_10}}
\end{figure}


\begin{figure}[!htb]
	\centering
	\includegraphics[width=1.0\linewidth]{chap1/1_11}
	\caption{717,527 名受试者超过 6800 万天的智能手机活动数据揭示了 111 个国家/地区的体力活动差异。
		图片改编自 Althoff 等人(2017 年)的研究,这是全球规模最大的体力活动调查\cite{althoff2017large}。 \label{fig:1_11}}
\end{figure}


在实验室或诊所进行实验时,除了运动捕捉系统外,还可能涉及多种专用设备。
我们经常使用测力板来测量脚和地面之间的力。在步行和跑步研究中,使用跑步机很方便,因为受试者可以保持在运动捕捉系统能够精确测量的范围内。
一些跑步机还配备了测量地面反作用力的仪器。
肌电图用于测量各种肌肉活动的时间和强度。
我们可以监测跑步者的呼吸,测量消耗的氧气量和产生的二氧化碳量,以估算跑步所需的代谢能量。
磁共振成像和荧光透视成像等成像技术的普及使我们能够看到运动中的动物或人体内部,为可视化和测量运动提供了强有力的工具。


概念模型可以成为强大的分析工具,正如我们将在本书中看到的。
例如,第~\ref{chap:chap2}~章和第~\ref{chap:chap3}~章展示了一个简单的摆模型如何为我们合理地模拟行走,而一个质量弹簧模型如何为我们提供关于跑步的重要见解。
一个能够完成任务的简单模型几乎总是比一个更复杂的模型更受欢迎,后者提供了类似的实用性,但构建和理解起来更困难。
当然,并非所有复杂现象都能用简单的力学模型来表示。
正如我们将在第~\ref{chap:chap11}~章和第~\ref{chap:chap12}~章中看到的,肌肉驱动的模拟是强大的工具,可以填补第~\ref{chap:chap2}~章和第~\ref{chap:chap3}~章中简单模型所缺失的许多细节。


计算机模拟可以计算无法直接测量的量并预测假设场景中的运动,从而对实验进行补充。
模拟有助于理解例如难以通过实验研究的损伤。我们还可以估算导致观察到的运动的肌肉力量,以及关节负荷、肌腱应变和其他无法测量的量。
在正向动态模拟中,我们规定一块或多块肌肉的神经激活模式,然后预测肌肉骨骼模型的最终运动(图~\ref{fig:1_12})。
需要实验数据来开发和测试用于运动模拟的肌肉骨骼动力学数学模型,并评估模拟反映现实的程度。


\begin{figure}[!htb]
	\centering
	\includegraphics[width=1.0\linewidth]{chap1/1_12}
	\caption{典型的正向动态模拟的要素。
		运动源于神经、肌肉、骨骼和感觉系统的复杂协调。
		这些系统的计算模型使我们能够预测和分析人类和动物的运动。 \label{fig:1_12}}
\end{figure}


当我们测量了测试对象的运动并希望将这些数据转化为有意义的见解时,通常会使用逆过程,例如,肌肉必须产生哪些力才能产生测量到的运动。
为此,我们需要进行逆动力学分析,这是一种将实验数据与肌肉骨骼模型相结合的常见分析策略。
第一步是使用身体的生物力学模型,将标记位置的测量值(如图~\ref{fig:1_10}~所示)通过称为逆运动学的过程转换为关节角度(图~\ref{fig:1_13})。
关节角度根据时间进行微分,以估计关节角速度和加速度,然后将其与施加于身体的外力测量值结合使用,以估计关节力矩。
然后,将生物力学模型与优化算法结合使用,以估计肌肉力量。
我们将这种策略称为逆分析,因为运动测量值可用于推断产生这些运动必须存在哪些力。


\begin{figure}[!htb]
	\centering
	\includegraphics[width=1.0\linewidth]{chap1/1_13}
	\caption{典型逆动力学分析的要素。
		分析从测量标记轨迹和外力(右)开始,并使用生物力学模型估算身体各节段和关节的角度、速度和加速度。
		逆动力学模型和优化程序可估算关节力矩和肌肉力量。 \label{fig:1_13}}
\end{figure}




\section{本书概述}

在接下来的章节中,我们将首先使用简单的概念模型研究人类两种常见的运动形式——行走和跑步。
然后,我们将通过研究骨骼肌的生物学和结构、其与肌腱的动态相互作用以及肌肉如何产生驱动骨骼的力量来探索运动的产生。
接下来,我们将研究用于分析运动的模型和算法。
我们将演示如何从运动捕捉数据和生物力学模型中计算关节角度、关节力矩​​和单个肌肉的力量。
最后,我们将综合这些概念来研究肌肉在行走和跑步过程中的作用,并就我对该领域未来发展方向的一些看法进行总结。
如图~\ref{fig:1_14}~所示,本书的内容被安排成四个部分,我鼓励读者以这种方式来理解本书。
因此,第~\ref{chap:chap4}~章(第二部分的开头)并非第~\ref{chap:chap3}~章的续篇,但第~\ref{chap:chap5}~章无疑延续了第~\ref{chap:chap4}~章的篇幅。
如果您记住这一点,本书的内容将对您更有意义。


% locomotion 通常是自主的(voluntary)运动
% movement: 移动(自主+非自主)
\begin{figure}[!htb]
	\centering
	\includegraphics[width=1.0\linewidth]{chap1/1_14}
	\caption{本书的组织结构。 \label{fig:1_14}}
\end{figure}



虽然我们主要关注人类的运动,但我们所描述的基本概念也可用于理解动物和机器人的运动。
本书涵盖的内容将帮助您理解精彩纷呈、内容丰富的科学文献,这些文献对众多主题进行了详尽的分析,而一本书根本无法涵盖所有​​这些主题。


\section{运动语言}

我想解释一下你需要了解哪些知识才能充分利用本书。
汤姆和我为熟悉某些工程基础知识的读者编写了本书。
数学和力学为分析运动提供了精确的框架,我们假设读者对向量和矩阵有基本的了解。
我们进一步假设读者熟悉自由体运动图、推导运动方程以及如何求解简单系统的运动方程。
如果你不熟悉这些主题,你应该准备好在遇到它们时花一些额外的时间去学习。


在生物学方面,了解人体解剖学和生理学背景会有所帮助,但并非必需。
对于不熟悉解剖学的读者,以下图表展示了本书将要用到的术语。这些术语包括解剖平面和方向(图~\ref{fig:1_15})、关节运动(图~\ref{fig:1_16}~和图~\ref{fig:1_17})以及主要骨骼和肌肉(图~\ref{fig:1_18}~和图~\ref{fig:1_19})。
这些术语乍一看可能令人望而生畏,但花几分钟时间研究这些图表并学习这些术语,对你大有裨益。


\begin{figure}[!htb]
	\centering
	\includegraphics[width=1.0\linewidth]{chap1/1_15}
	\caption{人体的解剖平面和方向。 \label{fig:1_15}}
\end{figure}


\begin{figure}[!htb]
	\centering
	\includegraphics[width=1.0\linewidth]{chap1/1_16}
	\caption{肩部、肘部、骨盆和臀部在冠状面(左)、矢状面(中)和横切面(右)的运动。 \label{fig:1_16}}
\end{figure}


\begin{figure}[!htb]
	\centering
	\includegraphics[width=0.75\linewidth]{chap1/1_17}
	\caption{膝盖和脚踝在矢状面上的运动。 \label{fig:1_17}}
\end{figure}


\begin{figure}[!htb]
	\centering
	\includegraphics[width=0.95\linewidth]{chap1/1_18}
	\caption{人体下肢的主要骨骼、解剖标志和肌肉(前视图)。 \label{fig:1_18}}
\end{figure}


\begin{figure}[!htb]
	\centering
	\includegraphics[width=0.75\linewidth]{chap1/1_19}
	\caption{人体下肢的身体部分和主要肌肉(后视图)。 \label{fig:1_19}}
\end{figure}


这本书只是我的一个开端,我希望它能成为你持续探索的旅程。
我们的梦想是,你能在此汇集的素材基础上,迸发出你独特的创造力火花,探索自然,创造一些能够丰富他人生活的东西。

















\chapter{行走} \label{chap:chap2}

\begin{figure}[!htb]
	\centering
	\includegraphics[width=0.5\linewidth]{chap2/2_0}
	% 加星号(*)表示不加编号
	\caption*{ \label{fig:2_0}}
\end{figure}

每走一步,你都会向前倾斜一点点,然后站稳,避免摔倒,一遍又一遍,你都在摔倒,然后站稳,避免摔倒。

\begin{flushright}
	——劳里·安德森 \\
\end{flushright}

\begin{figure}[!htb]
	\centering
	\includegraphics[width=0.5\linewidth]{chap2/2_0_2}
	% 加星号(*)表示不加编号
	\caption*{ \label{fig:2_0_2}}
\end{figure}

“有一天,我在月球上漫步,”
宇航员哈里森$\cdot$施密特兴高采烈地唱着,他即将踏上最后一次阿波罗任务的首次月球行走。
“在快乐的五月,”指挥官吉恩$\cdot$塞尔南附和道。
一缕缕月尘从他们的靴子上溅起,他们……究竟在做什么?跳跃?腾跃?跌跌撞撞?……穿过月球表面。


不管它是什么,它与我们通常认为的行走几乎没有什么相似之处。
施密特像个蹒跚学步的孩子一样慢步走着,左右摇晃。
塞尔南的步态看起来像个孩子骑着扫帚,却假装那是一匹马。
这两位训练有素的宇航员仿佛忘记了人类最基本的技能——行走——不得不学习一种新的移动方式(图~\ref{fig:2_1})。


\begin{figure}[!htb]
	\centering
	\includegraphics[width=1.0\linewidth]{chap2/2_1}
	\caption{宇航员很少在月球表面“行走”,他们更喜欢在月球引力下跳跃行走。
		图片由NASA提供。 \label{fig:2_1}}
\end{figure}


在本章中,我们将探讨为何看似简单的行走——人类进化已完美适应地球引力——在月球上却变得如此困难。
宇航员奇特的步态或许能帮助我们更充分地理解行走过程中发生的一系列精心安排的事件,以及引力在其中扮演的关键角色。


欣赏行走壮举的另一种方式是建造一台能够行走的机器。
塔德$\cdot$麦吉尔(Tad McGeer)的巧妙实验表明,一个拥有类似人类比例的无动力装置可以在倾斜的表面上行走。
这些装置无需大脑、脊髓或肌肉即可行走,只需在正确的方向上轻轻推一下即可。
这一观察表明,我们可以通过简单的机械模型来深入了解行走,我们将在本章中对此进行演示。


然而,在开始分析之前,有必要先了解一下步态周期以及运动中涉及的一些基本物理知识。
下一节将为你提供一些正确的方向。


\section{步行步态周期}

人类有两种常见的步态:行走和跑步。
我们都熟悉身体各部分在行走时所经历的典型周期性模式,但更正式地描述这些定性观察结果会很有帮助。
一个行走步态周期由同一条腿上连续两次的足部接触事件界定,另一条腿的足部接触通常发生在中途(图~\ref{fig:2_2})。
每条腿都有一个支撑期(此时足部接触地面)和一个摆动期(此时足部离地)。
支撑期始于足部接触地面,结束于足尖离地,对于一条腿而言,它占行走步态周期的约 60\%;其余时间则用于摆动。
由于支撑期比摆动期长,因此在每个行走步态周期中,都有双脚接触地面的时期,我们称之为双支撑期。
我们将只有一只脚接触地面的间隔称为单支撑期。


\begin{figure}[!htb]
	\centering
	\includegraphics[width=1.0\linewidth]{chap2/2_2}
	\caption{步行步态周期及其组成事件(例如,脚接触)和阶段(例如,双支撑)。 \label{fig:2_2}}
\end{figure}


步长是指两个连续足迹上同一点之间沿行进线的距离(图~\ref{fig:2_3})。
连续两步所走的距离,或一个步态周期所走过的距离,称为步长。足部接触事件发生的速率(相当于步长持续时间的倒数)称为步频或步频;
迈步的速率称为步频。
步行速度可以用步长与步频的乘积来计算,或者也可以用步长与步频的乘积来计算:


\begin{figure}[!htb]
	\centering
	\includegraphics[width=0.8\linewidth]{chap2/2_3}
	\caption{水平(地面)平面上的步态测量。 \label{fig:2_3}}
\end{figure}

\begin{equation}
	\text{速度} = \text{步长} \times \text{步频}
			    = \text{步长} \times \text{起落}
\end{equation}

个人通常的步行速度会因身高、体能和其他因素而异。
健康成年人在平地上通常选择以约1.2-1.4米/秒的速度行走,步频为2步/秒。
典型的步长约为0.6-0.7米。


另外两个值得注意的指标是在水平面上测量的(图~\ref{fig:2_3})。
步宽是脚平放时脚后跟中点与另一条腿上相同点之间的距离,垂直于进展线测量。健康成年人的步宽约为腿长的 10\%。
如果腿长约为 1 米,步宽约为 10 厘米,但对于正在学习走路的幼儿和一些平衡能力较差的人来说,步宽会更大。
足部进展角是进展线与连接脚后跟中点和第二个脚趾(即足部的长轴)的线之间的角度。
正和负的足部进展角分别称为外倾角和内倾角。
成年人的外倾度通常较小,约为 10 度,但这个值在患有肌肉骨骼或神经系统疾病的个体之间可能会有所不同。
例如,患有小脑性共济失调(会导致平衡障碍)的人可能会以更大的脚趾向外和步宽行走,以减少跌倒的风险。


\section{地面反作用力}

我们通过测量双脚与地面之间的力来研究行走(图~\ref{fig:2_4})。
我们将在第~\ref{chap:chap11}~章中学习更多关于行走过程中肌肉协调的知识;
目前,只需知道肌肉通过产生力来产生运动即可。
肌肉产生的“作用力”会导致地面对双脚施加“反作用力”。
行走过程中,可以使用测力板测量地面反作用力。
测力板是一种仪器,可以测量人在测力板上行走时在垂直方向、前后方向和左右方向受到的力。​​
地面反作用力很重要,因为它可以衡量身体重心在每个时刻的加速度。
我们可以使用牛顿第二定律将地面反作用力和其他外力与身体重心 (com) 的加速度联系起来:
\begin{equation}
	F_{\text{external}} - mg = m a_{\text{com}} \label{eq:2_2}
\end{equation}
其中 $F_{external}$ 是施加于身体的所有外力之和,$m$ 是身体的总质量,$g$ 是重力加速度,$a_{com}$ 是质心加速度。
值得花点时间思考一下上一句中“和”和“全部”这两个词的含义。
当双脚接触地面时,我们必须将施加于每只脚的力相加。由于公式~\ref{eq:2_2}~是矢量和,所以方向很重要。
双脚下方力的垂直分量支撑着身体的重量。
但是,正如您在图~\ref{fig:2_5}~中所看到的,前脚上的前后力往往会抵消后脚上的前后力;
也就是说,前脚充当了刹车的作用,阻止我们走得越来越快,而后脚则提供推进力。
如果没有施加外力(即 $F_{external} = 0$,则物体处于自由落体状态,其质心将以 $g = 9.81 m/s^2$ 的速度加速向地面坠落。



\begin{figure}[!htb]
	\centering
	\includegraphics[width=1.0\linewidth]{chap2/2_4}
	\caption{以 1.55 米/秒的速度行走时代表性的地面反作用力。
		图中显示了步态周期内的垂直和水平(前后)分量(左)以及总矢量示意图(右)。
		正水平力指向前方。
		较小的左右力未显示。
		数据来自 Dembia 等人(2017)。 \label{fig:2_4}}
\end{figure}


\begin{figure}[!htb]
	\centering
	\includegraphics[width=0.9\linewidth]{chap2/2_5}
	\caption{以 1.55 米/秒的速度行走时,用两块测力板测量地面反作用力。
		两组箭头表示每只脚随时间推移受到的力。​​
		如黑色箭头所示,在双脚支撑时,地面反作用力的前后分量指向相反的方向。 \label{fig:2_5}}
\end{figure}


行走过程中地面反作用力的记录显示出几个有趣的特征(图~\ref{fig:2_6})。
地面反作用力的垂直分量在足部接触地面后迅速上升,并在步态周期的约 10\% 处达到大于体重的力。
在站立中期,垂直力降至低于体重,然后在蹬地时再次上升至高于体重。
然后,垂直力降至零,因为在脚趾离地后,足部不再接触地面。
平均而言,总垂直地面反作用力等于体重的 1 倍。
请注意,根据公式~\ref{eq:2_2},当地面反作用力的垂直分量等于体重时,重心没有净垂直加速度,恰好平衡了重力产生的向下力。

\begin{figure}[!htb]
	\centering
	\includegraphics[width=1.0\linewidth]{chap2/2_6}
	\caption{以 1.55 米/秒的速度行走时,足部受到的典型地面反作用力。
		此处显示的力矢量(在空间中)与图~\ref{fig:2_4}~中显示的力矢量(随时间变化)相同。 \label{fig:2_6}}
\end{figure}


在站立的前半段,水平地面反作用力指向身体后部,使重心减速,之后指向身体前部。
除了图~\ref{fig:2_4}~所示的前后分量外,水平地面反作用力还有一个内外分量,这个分量虽然很小,但对于控制左右平衡很重要。
虽然在任何特定时刻重心都可能加速,但在几步匀速行走中,平均前后加速度为零。
改变前进速度时,平均前后加速度将不为零。
在垂直方向上,改变坡度时会出现非零平均加速度,例如从平地走到斜坡上时。


请记住,在步态周期的某些阶段,双脚都会接触地面,两个力相加,形成作用于身体的净向上力(图~\ref{fig:2_5})。
一个地面反作用力指向上方和前方,另一个指向上方和后方。
净地面反作用力主要指向上方,在双脚支撑阶段开始时有一个较小的向前分量,在脚趾离地时有一个较小的向后分量。
这些力抵消了向下的重力,并帮助您调节步行速度。


可以将测力板记录除以身体总质量,以估算质心加速度(公式~\ref{eq:2_2}~中的 $a_{com}$)。
该加速度可以积分一次以估算质心速度 ($v_{\text{com}}$),积分两次以估算其位置 ($r_{com}$)。
根据这些量,可以估算出质心向前的动能 ($E_{kf}$),其公式如下:
\begin{equation}
	E_{kf} = \frac{1}{2} m v_{\text{com}}
		   = \frac{1}{2} m 
		   	 ( \int a_{\text{com},f} (t) dt )^2  \label{eq:2_3}
\end{equation}
%
其中“$f$”下标表示前向分量
(请注意,垂直方向的速度波动非常小,因此我们在此忽略它)。
公式~\ref{fig:2_3}~中的积分符号提醒我们,加速度的影响是累积的,如果我们想要知道速度或动能,就必须将其随时间积分。
前向动能在站立中期最低,因为在站立的前半段,水平地面反作用力向后,使重心减速。
重力势能可以估算为:
\begin{equation}
	E_{\text{pg}} 
		= m g r_{\text{com},\text{v}} \label{eq:2_4}
\end{equation}
%
其中“v”下标表示垂直分量。
重力势能在站立中期达到最高,此时身体跨过站立肢(图 2.7)。
如果我们绘制步态周期中的前向动能和重力势能,我们会发现它们是异相的(图 2.8):
当重力势能接近最小值时,前向动能达到峰值,反之亦然。
因此,总能量几乎保持不变。
我们在行走过程中保存能量的方法之一是用重力势能换取前向动能或前向速度,类似于在重力影响下摆动的钟摆的情况。
这一观察结果表明重力与行走有很大关系。


\section{弹道步行模型}

1980年,托马斯·麦克马洪(Thomas McMahon)和西蒙$\cdot$莫雄(Simon Mochon)基于以下假设,开发了一个行走数学模型:身体将势能转化为动能,并最大限度地减少肌肉在摆动阶段的作用。
该模型非常简单,仅由三个刚性连杆和三个旋转(销)关节组成,值得注意的是,没有肌肉或运动(图2.9)。
支撑肢由一个倒立摆表示,该摆的踝关节固定在地面上,使肢体能够在矢状面上绕踝关节旋转;
支撑肢的膝盖被假设锁定在完全伸展的位置。
摆动肢被建模为一个双摆,大腿和小腿部分在膝盖处通过销连接。
两条腿在臀部处固定在一起,并具有真实的质量分布。
头部、手臂和躯干的质量被忽略,但它们的质量集中在臀部。


肌肉活动的实验记录表明,在正常速度行走过程中,除了在摆动期的开始和结束时,摆动肢的肌肉相对不活跃。
因此,弹道步行模型假设肌肉完全被动地确定模型启动和完成摆动期所需的肢体节段的位置和速度,仅受重力作用(就像抛射物,因此称为“弹道”)。
当肌肉不活跃时,摆动肢的行为被认为类似于非受迫双摆。
如果在脚趾离地时为模型提供恰到好处的初始条件,则摆动肢的脚趾将在摆动中期离开地面,并且膝盖将在脚接触时完全伸展。


\section{弗鲁德数}

倒立摆本身是一个比弹道行走模型更简单的模型,它无法准确预测人类的步行速度,但确实提供了一些关于人类步行速度的物理限制的重要见解。
考虑一个倒立摆,如图 2.9 左侧所示,长度为 $l$,体重 $m$ 集中在臀部。
当臀部处于最大高度时,其瞬时速度 $v$ 水平方向,踝关节的垂直反作用力 ($F$) 等于向心力:
%
\begin{equation}
	F = \frac{m v^2}{l}
\end{equation}
% 
向心力是必须施加在摆锤上,以防止其离开地面的向下力。
由于地面无法对脚施加向下的力(除非你踩到胶水),所以向心力必须由体重 $mg$ 产生。
将这些力相等,并解出速度,我们发现倒立摆模型的最大行走速度 ($v_{max}$) 为
%
\begin{equation}
	v_{\text{max}} = \sqrt{g l}
\end{equation}

我们用$v_{max}$来定义无量纲步行速度($v^{*}$):
\begin{equation}
	v_{*} = \frac{v}{v_{max}}
\end{equation}
%
当速度超过 $v_{max}$ 时,脚会离开地面;因此,$0 \leq v^{*} \leq 1$,此时我们行走时就像一个倒立摆。
弗劳德数 ($F_r$) 是一个无量纲量,表示向心力与重力之间的比率:
%
\begin{equation}
	F_r = \frac{v^2}{g l}
		= (v^{*})^2
\end{equation}
%
无量纲步行速度和弗劳德数为比较不同最大速度和腿长的动物和人的步行速度提供了非常有用的指标。


倒立摆模型的一个重要预测是,如果 $l$ 或 $g$ 减小,最大步行速度也会减小。
当观察一个孩子与一个成年人并肩行走时,第一个关系很明显。
由于腿较短,孩子的弗劳德数会比成年人高,因此 $v_{max}$ 会较低。
因此,您可能会观察到成年人以舒适的速度行走,而孩子则跑着跟上。 
$v_{max}$ 对重力 $g$ 的依赖性解释了宇航员在月球表面遇到的挑战。
由于月球上的重力仅为地球的六分之一左右,因此月球上的 $v_{max}$ 与地面上的值相同。
因此,正常的地球步行速度会导致宇航员离开月球表面。
因此,阿波罗 11 号宇航员尼尔$\cdot$阿姆斯特朗和巴兹$\cdot$奥尔德林大多以慢速行走,以保持双脚着地。
后来的宇航员采用了各种各样的跳跃步态。
有趣的是,登月任务前的实验表明,袋鼠跳是最有效的步态,但宇航员很少采用这种策略。



值得注意的是,人类和其他陆地动物通常不会在弗劳德数为 1 时行走;
它们会在弗劳德数约为 0.5 时从行走过渡到奔跑以节省能量。
一个重要的例外是大象,它是最大的陆地动物,当弗劳德数大于 1 时,它们的行走速度似乎比倒立摆模型允许的速度快得多 (Hutchinson 等人,2003)。
与往常一样,理论与观察之间的差异提供了学习的机会。
从 2003 年到 2010 年,John Hutchinson 及其同事研究了视频,并使用定制的测力板对大象进行了实验。
(它们必须是强大的测力板!)
Hutchinson 发现,尽管大象始终至少有一只脚着地,因此仍然是传统意义上的“行走”,但它在高速下会夸张地蹲下(图 2.10)。
这种“格劳乔步态”以喜剧演员格劳乔$\cdot$马克斯(Groucho Marx)的名字命名,他推广了这种行走方式。
这种步态意味着腿部肌肉产生巨大的力量,重力势能和动能的交换与倒立摆行走模型不一致。
因此,大象的步态某种程度上是行走和跑步的混合体。
下一章将讨论行走和跑步的区别,以及另一种大型动物——霸王龙——是否能够奔跑,届时我们将更详细地解释这一概念。


\section{运输成本}

弹道步行模型捕捉到了正常步行的一些显著特征,部分原因是我们自然而然地学会了如何以最小化能量消耗的方式移动。
减少腿部摆动时的肌肉活动就是一个例子。
我们也会自然地选择步行速度、节奏、步宽和其他变量,以最小化传输成本或移动给定距离所需的能量(图 2.11)。
过去 60 年来的许多研究已经证实,我们行走的方式可以最小化传输成本——这是生物力学的一个重要原则。
步行的传输成本通常是通过收集和分析一个人以特定速度行走时吸入和呼出的混合气体来估算的(图 2.12)。


请注意,我们无法直接测量能量消耗,但基于呼吸测量的间接估算是一个很好的替代方法。
我们的肌肉由燃烧碳水化合物、脂肪和蛋白质的化学反应提供动力。
这些反应消耗氧气并产生二氧化碳。通过确定废气产生的速率,我们可以推断出使用了多少燃料,从而消耗了多少能量。
如图 2.11 所示,我们通常将运输成本表示为每公斤体重移动一米所需的焦耳能量。



虽然弹道模型有助于理解行走的一些基本特征,但它也存在一些局限性。
例如,它无法模拟双支撑,而这对于理解连续步伐之间的过渡至关重要。
倒立摆模型预测我们的最大行走速度将发生在弗劳德数为 1 时,而实际上人类在弗劳德数远低于 1 时从行走过渡到跑步,竞走运动员在弗劳德数大于 1 时可以“行走”。
还要注意,弹道步行模型可能会让人得出结论,认为行走不会消耗任何能量。
此外,具有刚性站立肢的模型无法准确预测实验观察到的地面反作用力。
最后,弹道步行模型不会产生重复的步态周期。
其中一些限制在一个稍微复杂一些的模型(称为动态步行模型)中得到了解决。


\section{动态步行模型}

弹道行走模型在钟摆模型的基础上,通过增加膝盖(或者更准确地说,一个膝盖)进行了改进。
我们或许会认为,再增加一个解剖学元素就能进一步改进模型。
但它的作用远不止于此:
通过增加双脚,塔德$\cdot$麦吉尔设计了一个可以在实验室中制造和测试的行走机器人。


麦吉尔曾接受过航空工程师的培训,因此他采用的策略与一个世纪前飞机研发的策略如出一辙也就不足为奇了。
在尝试动力飞行之前,莱特兄弟多年来一直致力于研发利用滑翔机下坡时重力势能驱动的滑翔机。
到1902年底,他们已经完成了数百次此类飞行。
莱特兄弟掌握了滑翔技术后,便自信满满地掌握了动力飞行,并于次年完成了首次动力飞行记录(Collins 等人,2005)。


效仿莱特兄弟,在莫雄和麦克马洪提出弹道模型十年后,麦克吉尔构建了一种被动机构,在适当的初始条件下,该机构可以仅靠重力驱动,稳定地从缓坡上行走(图 2.13)。
证明这种机构的可构建性是一项突破,并开启了“动态行走”研究的新领域,动态行走主要由腿部的被动动力学产生运动。
如今,动力驱动的动态步行机器人的设计遵循莱特兄弟的原理:
如果被动机构能够严格在重力作用下从缓坡上移动,那么主动机构应该能够使用执行器在平地上移动,执行器仅注入重力在下缓坡时提供的少量能量。



动态步行模型在三个关键方面扩展了弹道模型:添加双脚、建模双支撑,以及最重要的,实现一步一步的过渡(图 2.14)。
这些特性使该机构能够进行周期性、连续的步行。
在单支撑期间,支撑肢的脚在地面上滚动(不滑动),而摆动肢被动摆动。
膝关节伸展止动装置可防止摆动结束时膝关节过度伸展,并保持支撑肢完全伸展,被动支撑身体重量。
在支撑的前半段,重心向上移动,在后半段向下移动。
就在脚接触之前,支撑肢踝部产生的推离力矩将重心重新定向到向上的轨迹。
在脚接触之前重新定向重心可降低脚与地面碰撞的速度和相应的能量损失。
在脚与地面碰撞时以及膝盖完全伸展时撞到止动装置时,仍然会损失一些能量。
为了实现连续的步态,每一步都必须注入少量能量,以补偿能量的耗散以及关节的摩擦损失。
在完全被动的步行机中,这种能量由行走机构在缓坡下行时产生的重力势能提供;
在平地上,这种能量可以由位于踝部或臀部的电机注入。


动态步行模型为理解平地行走的一些基本特征提供了一个理论框架。
动态步行模型已用于研究步态周期中的能量消耗、能量消耗如何随步行速度而变化以及人类步行的其他方面。
例如,在开始双脚支撑时改变质心轨迹所需的能量称为步进转换成本。
动态步行模型正确地预测了该成本会随着步幅更大、速度更快而增加。
增加步长(同时保持步频恒定)会增加能量消耗,因为质心速度增加,其轨迹必须改变更大的幅度。
增加步频(同时保持步长恒定)也会增加能量消耗,因为腿部必须比单独由被动动力学产生的速度更快地摆动,而肌肉会消耗能量来摆动腿部。


步步过渡和腿部强制摆动运动所产生的能量消耗,除了维持平衡外,还会影响人类行走的成本。
人类通常会选择步长和步频,以最小化运输成本。
当步行速度超过运输成本最低的速度时,人类会以几乎等比例增加步长和步频,以平衡步步过渡和腿部强制摆动带来的成本增加。
步步过渡的成本也可以通过将我们的双脚像轮子的一部分一样使用(请注意图 2.14 下方的弧形双脚)来降低,从而减少所需的重心方向变化。
当地面反作用力从脚接触时的后脚掌转移到脚趾离地时的前脚掌时(图 2.5),人类的双脚实际上在地面上滚动。
对于动态步行者来说,脚的位置至关重要:
如果脚的位置过于前倾,步行者就会向后摔倒;
如果腿部摆动速度过慢且步长过短,步行者就会向前摔倒。


动态步行模型比弹道模型具有更强的分析能力,但当然也存在局限性。
许多建模假设和简化使得该模型无法用于研究人类步行的某些要素。
例如,动态步行模型假设运动严格为平面运动,忽略了矢状面以外的运动。
McGeer 的装置通过将腿部数量增加一倍来最大限度地减少左右摇摆,从而强化了这一假设。
2001 年,当时就读于康奈尔大学的史蒂夫·柯林斯 (Steve Collins) 制造了第一台双足被动动态步行机(图 2.15)。
设计这种机制需要仔细关注机器质心的轨迹,以防止其因非矢状面运动而跌倒。
矢状面上的反向摆动臂稳定了偏航,足部形状和侧向摆动臂控制了倾斜,而柔软的鞋跟则避免了对足部接触时姿势的敏感性(Collins 等人,2001)。
在每种情况下,仿生学都改善了机制的性能,并且让我们了解了为什么这些要素(手臂摆动、脚和脚跟)会给人类带来生物力学优势。


\section{手臂摆动}

到目前为止,我们已经使用了简单的模型来研究下肢的动力学,但上肢在行走中也发挥着作用。
上肢最明显的运动是手臂摆动,但我们行走时为什么摆动手臂却不那么明显。
为了研究这种行为,史蒂夫$\cdot$柯林斯和他的同事提出了一种类似于麦吉尔的直腿被动行走机制,但在臀部连接了一个类似手臂的摆锤(图 2.16)。
他们测试了几种手臂摆动策略,包括我们熟悉的“正常”摆动,以及一种“反正常”摆动,即左臂随左腿前进,右臂随右腿前进。
我不禁将这种策略称为“兄弟步”,以纪念我的哥哥布莱恩$\cdot$德尔普,他小时候喜欢这样走路,我们俩都喜欢。


对正常行走的人类受试者进行的实验表明,肩关节和肘关节肌肉的活性较低,这证实了正常的手臂摆动主要是被动的。
下肢的运动学和动力学不受手臂摆动策略的影响,但当手臂不摆动时,整个身体会绕垂直轴旋转得更大,而在反常手臂摆动时,旋转幅度更大。
角动量的增加被绕垂直轴更高的地面反作用力矩所抵消,从而导致肌肉活性和能量消耗相应增加。
简而言之,Bro步法很不协调,这就是为什么超过一定年龄的人通常不会这样走路。


\section{用于步态分析的骨骼模型}

由于被动步行器的每个仿生特征都提升了其性能,人们自然会好奇一个更高保真度的模型会是什么样子。
让我们先来听听一些坏消息:
解剖关节非常复杂,相邻的身体部位会在各个方向上相对平移和旋转。
我们通常只关注这些运动中的一小部分,或者只能精确测量它们。
因此,即使是刻意模仿人体的模型也会做出一些简化的假设,例如不允许股骨头和骨盆之间的相对平移,从而将髋关节表示为球窝关节。


尽管如此,图 2.17 展示了一个用于分析步态的典型且相当准确的下肢骨骼模型。(必要时可添加躯干和手臂。)
该模型由 9 个关节刚体组成:骨盆、左右股骨、髌骨、胫腓骨(小腿)和足部。
该模型的下肢有 16 个自由度:6 个自由度描述骨盆相对于固定参考系的位置和方向(倾斜、侧倾和旋转),5 个自由度描述每条腿的姿势(髋屈曲、内收和旋转;膝关节伸展;踝关节背屈)。
这些术语是标准术语,值得学习,如图 1.16 所示。


想一想:在我们行走的每一刻,我们都在不知不觉中同时控制着十几个角度位置。
相比之下,莫洪和麦克马洪的弹道行走模型只有3个自由度,动态行走模型也只有几个自由度。
我们之所以成功,部分原因在于我们走捷径,尽可能地使用被动运动,但也因为我们的大脑在生命的第一年花费了大量时间学习如何协调复杂的行走动作。
下次你再开玩笑说有人不能一边走路一边嚼口香糖时,请记住,即使是第一步——双足行走——也是一项了不起的成就。


请注意,图 2.17 所示的模型是三维的。
正如我们所见,行走是一种三维活动,必须分析非矢状面的运动和力才能理解平衡和体重支撑。
有多种方法可以测量这些三维运动,我们将在第 7 章中介绍。
我们可以使用这些方法来估计人类受试者在行走过程中的运动,我们将在下文中介绍。


\section{步行运动学}

行走时,骨盆会经历复杂的运动(图 2.18)。
在额状面上,由于肢体在站立初期负重,骨盆会向摆动侧向下倾斜。
该运动的范围几乎翻倍,从低速行走时的约 5 度增加到高速行走时的约 10 度。
在横切面上,骨盆向前进肢旋转;
因此,前肢的髋部位于后肢的髋部前方。
该运动也会随着速度的增加而增强,并提供了一种增加步长的机制。


下肢关节也以典型的模式运动(图2.19)。
足触地时,髋关节处于屈曲状态。
在站立期,髋关节伸展,在足尖离地前达到最大伸展度,然后在摆动期屈曲。
足触地时,膝关节完全伸展。
当肢体负重时,膝关节像减震器一样先屈曲后伸展,速度越快,屈曲度越大。
在摆动前,膝关节快速屈曲,在摆动中期附近达到最大屈曲度,然后快速伸展,在下一次足触地前达到完全伸展。
足触地时,踝关节处于中立位。
在站立初期,随着足部向地面旋转,踝关节跖屈;
当胫骨越过足部时,踝关节背屈。
在站立期接近尾声时,踝关节快速跖屈,大约在足尖离地时达到最大伸展度。
随着步行速度的增加,峰值关节角度通常会增加,而步态周期中站立的时间会减少。


\section{地面反作用力和步行速度}

图 2.20 显示了行走过程中的地面反作用力。
高速行走时,地面反作用力的垂直分量在足部触地后迅速上升,并呈现出特征性的双峰形状。
我们在图 2.4 至 2.6 中也看到了类似的形状。
随着速度的增加,这些峰会变得更加明显。
地面反作用力的第一个峰来自支撑身体重量的前肢肌肉。
地面反作用力的第二个峰来自蹬地时后肢肌肉。
这些反作用力有助于我们了解重心的加速度,但仅凭实验无法告诉我们哪些肌肉负责产生测量到的地面反作用力。
我们将在第~\ref{chap:chap11}~章中研究肌肉如何协调行走并产生地面反作用力。


图 2.18–2.20 所示的数据可从 simtk.org 免费下载。
此类规范数据对于量化受试者的步态偏差以及测试运动模型和模拟的准确性非常有价值。


\section{非典型步态}

人类可以利用本章讨论的机制平稳高效地行走。
然而,身体或大脑的损伤以及关节或肌肉的疾病可能会扰乱行走动力学。
例如,脑瘫患者(一种因脑损伤引起的运动障碍)经常以蹲伏步态行走(图 2.21)。
蹲伏步态的特点是站立期膝关节过度屈曲。
膝关节过度屈曲会带来问题,因为它会增加站立期膝关节的力量,阻碍摆动时脚趾与地面的间隙,并显著增加能量消耗(尝试以蹲伏步态行走两分钟,看看是否会气喘吁吁)。
脑瘫患者的膝关节过度屈曲通常会随着时间的推移而恶化,常常导致膝关节力学改变和慢性膝关节疼痛。
在严重的情况下,膝关节屈曲程度可能会变得非常严重,以至于患者完全丧失行走能力。


许多脑瘫患者以及中风患者行走时,都会出现膝关节僵硬的步态,即摆动期膝关节屈曲功能减弱且延迟(图 2.22)。
这种步态还会降低脚趾与地面的距离,导致绊倒或需要进行能量效率低下的代偿性运动。
膝关节僵硬的步态被认为主要是由股直肌活动不当引起的,股直肌经髌骨穿过膝关节前方,产生膝关节伸展力矩。
因此,膝关节僵硬的步态通常采用股直肌转移术治疗,将股直肌的附着点从髌骨转移到一个能够降低其产生膝关节伸展力矩能力的部位。
遗憾的是,股直肌转移术的结果并不一致:
有些人在手术后摆动期膝关节屈曲功能有显著改善,而另一些人则几乎没有变化。


仅仅通过检查关节运动学,无法理解蹲伏步态或膝僵硬步态的成因,因为运动测量并不能确定导致该运动的原因。
正如我们将在第~\ref{chap:chap11}~章中看到的,肌肉驱动的步行模拟可以为理解蹲伏步态和膝僵硬步态的成因提供参考,并可用于设计有效的治疗方法。


\section{不同条件下步行的变化}

步行是一项典型的活动,这意味着我们可以从简单的模型和平均实验数据中了解到很多相关信息。
然而,并没有单一的理想步态。我们都有过仅凭走路姿势就能认出远处朋友的经历。
此外,步行方式也有很多变化,这些变化完全是针对不同情况的“正常”适应。
人们在走得慢或快、搬运杂货、爬山、参加游行或穿着高跟鞋时,步态都会有所不同。
我们还可以观察到,当佩戴经过优化以降低步行能量消耗的机器人系统时,步行动力学的变化。
表 2.1 列出了在各种条件下步行时观察到的一些变化,除下坡步行外,所有这些变化都会增加运输成本。
当然,在稳态条件下以及步态启动、转弯和加速等瞬态条件下,步行过程中还观察到了更多变化。


% 表


人类是高效的步行者。
在最好的情况下,我们只需向前跌倒,然后迈步来避免跌倒,并注入少量能量来过渡步伐并摆动双腿。
然而,许多情况下,步行功能受损,这可能会限制日常生活活动。
步行是数十亿人的主要身体活动形式,而身体活动受限会带来严重的健康后果。
为了帮助神经系统和肌肉骨骼疾病患者恢复和改善步行能力,深入了解步行动力学至关重要。
本章只是触及皮毛。
稍后,我们将探讨正常和受损步行状态下肌肉的活动。
现在,我们先从步行过渡到跑步。














\chapter{跑步} \label{chap:chap3}

如果你能用六十秒的长跑来填补这无情的一分钟……
\begin{flushright}
	——拉迪亚德$\cdot$吉卜林
\end{flushright}


\begin{figure}[!htb]
	\centering
	\includegraphics[width=1.0\linewidth]{chap3/3_0}
	% 加星号(*)表示不加编号
	\caption*{ \label{fig:3_0}}
\end{figure}

看过《侏罗纪公园》的人可能都记得那个标志性的场景:
一辆吉普车试图超越一只追赶的霸王龙。
有趣的是,这个场景与电影上映时(1993年)古生物学家的普遍看法相符。
当时人们认为霸王龙的奔跑速度可以达到 40 公里/小时,一些科学家甚至认为它的速度甚至可能更快。
以这样的速度,它在土路上完全可以超过一辆吉普车。


然而,自 2002 年约翰$\cdot$哈钦森(John Hutchinson)开创性论文以来,近期的生物力学研究表明,霸王龙可能根本无法奔跑。
即使它能跑,也可能无法达到 40 公里/小时的速度。
哈钦森通过模拟霸王龙,证明要达到这样的速度,其 86\% 的体重必须由腿部肌肉构成,留给其庞大的尾巴、头部和躯干的空间非常有限。
为了产生足够大的地面反作用力,它需要将如此大的质量分配给腿部肌肉,而这种反作用力在奔跑时通常超过体重的两倍。


对恐龙、大象和袋鼠等动物的研究有助于我们理清对人类跑步的理解:
是什么驱动着从步行到跑步的转变,又是什么限制了跑步速度。
地面反作用力的测量揭示了为什么跑步时比步行时更容易受伤,以及如何设计跑道来降低受伤率并提高速度。


在本章中,我们将使用简单的力学模型来探讨这些问题。
这些模型包含弹簧,用于表示肌肉和肌腱的弹性特性,并揭示弹性能量的储存和释放如何提高跑步效率。
首先,我们定义跑步步态周期,并研究其中涉及的力和弹性机制。
然后,我们将探索一些原理,帮助你设计跑道、跑鞋和假肢,从而实现快速高效的跑步。
我们还会研究从步行过渡到跑步时步态和能量消耗的变化。


\section{跑步步态周期}

跑步步态周期由单腿支撑和腾空交替的阶段组成(图~\ref{fig:3_1})。
与步行类似,一个跑步步态周期由同一条腿连续 2 次触地事件定义,其中对侧腿的触地发生在半程。
每条腿都有一个\textit{支撑期}(脚与地面接触)和一个\textit{摆动期}(脚离地)。
支撑期始于脚触地,结束于脚趾离地,在人类跑步过程中约占步态周期的 30\% 到 45\%,但在高速冲刺过程中可能只占 25\% 或更少。
回想一下,步行时支撑期的持续时间会随着步行速度的增加而缩短。
因此,随着步行速度的增加,双腿支撑的持续时间会缩短。
如果进一步提高速度,每条腿对应的支撑期最终将持续不到步态周期的一半,从而进入腾空期。


\begin{figure}[!htb]
	\centering
	\includegraphics[width=1.0\linewidth]{chap3/3_1}
	\caption{跑步步态周期及其组成事件(例如,脚部触地)和阶段(例如,支撑)。
		站立和摆动的时间百分比会随着跑步速度和风格而变化。 \label{fig:3_1}}
\end{figure}

腾空阶段的存在是我们区分行走和跑步的一个方式。
事实上,人们很容易相信它是跑步的特征,不仅对人类如此,对其他动物也是如此。
然而,我们很快就会看到,当我们从行走步态转换为跑步步态时,会发生另一种质的变化,从生物力学的角度来看,这一点更为重要。


用于量化跑步步态周期的指标与用于步行的指标类似。
步长是两个连续足迹之间沿行进线的距离。
连续两步所走的距离,或一个步态周期所走过的距离,称为步长。
足部接触事件发生的速率(相当于步长持续时间的倒数)称为步频;
迈步的速率称为步频。
跑步速度可以用步长与步频的乘积来计算,
或者,也可以用步长与步频的乘积来计算:

\begin{equation}
	\text{速度(米/秒)} = \text{步长(米/步)} \times \text{节律(步/秒)} \label{eq:3_1}
\end{equation}

中等跑步速度为 4 米/秒。
在此速度下,支撑期约占步态周期的 35\% 至 40\%,典型的步频为 180 步/分钟,典型的步长约为 1.3 至 1.4 米:

\begin{equation}
	\text{步长} = \frac{4.0 \text{米}}{1 \text{秒}}
				 \times \frac{60 \text{秒}}{1 \text{分钟}}
				 \times \frac{1 \text{分钟}}{180 \text{步}}
				 = 1.33 \text{米/秒}
				 \label{eq:3_2}
\end{equation}

当然,这些数量会随着腿长、跑步风格、鞋类和其他因素而变化。


\section{地面反作用力}

正如我们在第~\ref{chap:chap2}~章中所看到的,我们可以通过测量地面反作用力随时间的变化来深入了解步态的动力学和能量学。
跑步时,地面反作用力的垂直分量在足部触地后迅速上升,并在步态周期的约15\%到20\%处达到最大值(图~\ref{fig:3_2})。
在中等速度跑步时,垂直地面反作用力在约 2 倍体重时达到峰值。
水平地面反作用力在站立的前半段指向后方,使重心减速,之后指向前方。

\begin{figure}[!htb]
	\centering
	\includegraphics[width=1.0\linewidth]{chap3/3_2}
	\caption{跑步时足跟着地时代表性的地面反作用力。
		图中显示了步态周期内的垂直和水平(前后)分量(左)以及总矢量示意图(右)。
		垂直地面反作用力在足跟着地时出现一个急剧的峰值\cite{yong2020foot}。 \label{fig:3_2}}
\end{figure}

图~\ref{fig:3_2}~显示了后脚掌着地的跑步者(即用脚跟着地的跑步者)身上特有的双峰垂直地面反作用力。
虽然行走时的地面反作用力也有两个峰值,但跑步时第一个峰值较短,是由于脚跟着地产生的。
某些跑步者,尤其是那些从小就赤脚跑步的跑步者,会用前脚掌着地。
对于这些跑步者来说,垂直地面反作用力上升得更平缓,并且只有一个峰值(图~\ref{fig:3_3})。
有人认为这种跑步方式更“自然”,可以防止与冲击相关的伤害,但后续研究对这一结论提出了质疑。
从小就用后脚掌着地的跑步者如果改为用前脚掌着地,尤其是在没有经过适当训练的情况下,可能会有受伤的风险。

\begin{figure}[!htb]
	\centering
	\includegraphics[width=1.0\linewidth]{chap3/3_3}
	\caption{前脚掌落地时,跑步过程中地面反作用力的代表性数据,与图~\ref{fig:3_2}~中的受试者相同。
		垂直地面反作用力没有后脚掌落地时出现的尖峰,这可能有助于降低受伤风险\cite{yong2020foot}。 \label{fig:3_3}}
\end{figure}


跑步时,前向动能和重力势能可以用公式~\ref{eq:2_3}~和公式~\ref{eq:2_4}~计算(图~\ref{fig:3_4})。
在跑步的腾空阶段(忽略空气阻力),前向速度和前向动能最大,且近似恒定。
重心在腾空过程中也最高,因此重力势能也最高。
因此,与行走不同,前向动能和重力势能大约同时达到最大值。
同样,前向动能和重力势能都在站立中期达到最小值,也就是说,当重心最低时,前向速度大约最小。

\begin{figure}[!htb]
	\centering
	\includegraphics[width=1.0\linewidth]{chap3/3_4}
	\caption{奔跑过程中的代表性重力势能和前向动能。
		飞行过程中,前向动能保持不变\cite{yong2014differences}。 \label{fig:3_4}}
\end{figure}


我们想强调步行和跑步之间的一个根本区别:
在跑步时,我们并非从向前的动能和重力势能的交换中获益,而是在肌肉和肌腱的伸展和回弹过程中储存和释放弹性势能。
这一观察结果暗示了一种如图~\ref{fig:3_5}~所示的跑步模型。
在该模型中,质量在站立中期达到最低点,此时质心的向前速度也最小,这与我们在实验数据中发现的情况相似。
我们腿部如同弹簧般的运动使跑步成为一种能量高效的步态。

\begin{figure}[!htb]
	\centering
	\includegraphics[width=1.0\linewidth]{chap3/3_5}
	\caption{跑步的站立阶段(左)及其质量弹簧模型(右)。
		在该模型中,身体的质量被集中到一个点质量上,该点质量位于代表腿部的无质量线性弹簧的顶部。
		弹簧压缩,质量在站立中期达到最低点,此时质量的前进速度也最低。 \label{fig:3_5}}
\end{figure}


正如我们在第~\ref{chap:chap2}~章中看到的,大象即使在尽可能快地移动时也没有飞行阶段。
因此,你可能会认为大象不会跑。
但事实上,它们似乎有一种“格劳乔式行走”,膝盖弯曲,重力势能和向前动能的波动同步。
从生物力学的角度来看,这种同步意味着大象在某种意义上是在奔跑。
就像人类一样,这些厚皮动物利用腿作为弹簧来储存能量,如图~\ref{fig:3_5}~所示。
大象很可能进化出这种不腾空而起的“奔跑”方式,以适应它们巨大的体型。
正如我们将在本书后面看到的,肌肉力量与肌肉的横截面积(长度的平方)成正比,而体重与体积(长度的立方)成正比。
因此,体型巨大的动物与体型较小的动物相比,体型较大的动物具有较低的力量重量比。
这个原理解释了为什么像松鼠这样的小动物可以产生很大的地面反作用力并跳跃几个身长的距离,而大象和霸王龙(它们的大小大致相同)却无法产生足以让它们瞬间飞翔的地面反作用力。



\section{跳跃和跑步中的弹性机制}

另一种能够让我们了解跑步过程中弹性能量储存的动物是袋鼠。
袋鼠的奔跑并非传统意义上的奔跑;
我们通常将它们的动作描述为跳跃。
然而,从生物力学的角度来看,它们使用的机制与人类跑步时类似。


20世纪70年代初,特伦斯$\cdot$道森和理查德$\cdot$泰勒训练袋鼠佩戴面罩,在跑步机上以1至22公里/小时的速度跳跃——这项实验无疑需要高超的动物操控技巧。
面罩让研究人员能够测量动物的代谢能量消耗,而这正是泰勒当时研究的主题。
他针对袋鼠和其他​​动物进行的大量实验,让生物学家们注意到能量消耗是影响动物行为的关键因素。


道森和泰勒发现,低速时,袋鼠不会跳跃,而是采用五足步态,即用尾巴作为与地面的第五个接触点(图~\ref{fig:3_6})。
这种运动方式看起来很笨拙,而且能量效率低下。
事实上,他们发现,当袋鼠以更高的速度(从 1 到 6 公里/小时;图~\ref{fig:3_7}A)蹒跚向前时,能量消耗急剧增加。
五足步态下速度的提升主要通过增加步频实现(图~\ref{fig:3_7}D)。
当速度达到约 6-7 公里/小时时,袋鼠会过渡到跳跃。
图~\ref{fig:3_7}A~中一个引人注目的特征是,当袋鼠的速度超过 7 公里/小时时,运送成本略有下降。
如图~\ref{fig:3_7}C~和 D 所示,在跳跃阶段,它们主要通过增加步幅来提高速度。
步频几乎保持不变。这一结果与袋鼠的质量弹簧模型(类似于图~\ref{fig:3_5})一致,因为弹簧质量的固有频率不会随着其运动幅度的变化而变化。
Dawson和Taylor指出,袋鼠的跟腱非常适合储存和释放弹性能量,并认为这种机制使跳跃更加高效,并使其能够高速跳跃。


大约在这个时候,泰勒在意大利乔瓦尼$\cdot$卡瓦尼亚的实验室里了解到了测力板,于是他带着他的动物(袋鼠、猴子、狗和火鸡)来到米兰,在卡瓦尼亚当时刚刚发明的实验中试用它们。
最终的研究成果发表于1977年,进一步证明了袋鼠会储存弹性能,并在跳跃过程中释放。
在高达30公里/小时的速度下,氧气消耗约占袋鼠加速重心所需能量的三分之一;
剩余的能量必定是由肌肉和肌腱中的弹性储存提供的。


\begin{figure}[!htb]
	\centering
	\includegraphics[width=1.0\linewidth]{chap3/3_6}
	\caption{袋鼠的运动方式。
		在慢速运动时,袋鼠采用五足步态(上图),利用尾巴支撑后肢向前移动。
		在高速运动时,袋鼠采用跳跃步态(中图),利用尾巴保持平衡并控制身体倾斜。
		在高速跳跃时(下图),袋鼠的速度可以超过 50 公里/小时\cite{dawson1977kangaroos}。 \label{fig:3_6}}
\end{figure}


\begin{figure}[!htb]
	\centering
	\includegraphics[width=1.0\linewidth]{chap3/3_7}
	\caption{袋鼠运动的能量学。
		在五足步态中,质量标准化的耗氧率随速度增加而增加。
		袋鼠在6-7公里/小时左右过渡到跳跃;
		随后耗氧量随速度增加而下降,直至20公里/小时左右达到最低值。
		Dawson\cite{dawson1977kangaroos}估算了无法在实验室研究的速度下的耗氧量(虚线)。
		在低速跳跃时,步频保持相对恒定,速度主要通过增加步幅来提高\cite{dawson1977kangaroos}。 \label{fig:3_7}}
\end{figure}











\part{运动产生}
\markboth{运动产生}{运动产生}

\chapter{肌肉生物学和力量}\label{chap:chap4}


孤身一人,我们能做的太少。
团结起来,我们能做的却很多。
\begin{flushright}
	————海伦$\cdot$凯勒
\end{flushright}


\begin{figure}[!htb]
	\centering
	\includegraphics[width=1.0\linewidth]{chap4/4_0}
	% 加星号(*)表示不加编号
	\caption*{ \label{fig:4_0}}
\end{figure}


伦敦皇家学会拥有350年的历史,拥有近200年的传统,每年都会邀请科学家进行公开演讲,并经常进行一些简单的实验。
1952年,其中一项实验成为了热议话题。


两辆固定自行车朝向相反,并通过一条链条连接在一起,当一辆自行车向前踩踏板时,另一辆自行车的踏板就会向后移动(图~\ref{fig:4_1})。


\begin{figure}[!htb]
	\centering
	\includegraphics[width=1.0\linewidth]{chap4/4_1}
	\caption{Abbott 等人 (1952) 描述的推拉装置。 \label{fig:4_1}}
\end{figure}


一辆自行车上坐着一位娇小的女子,名叫布伦达$\cdot$比格兰,是肌肉疲劳研究领域的权威专家。
另一辆自行车上坐着一位魁梧的年轻人,名叫默多克$\cdot$里奇,他嫁给了他的自行车“对手”。
演讲者是A$\cdot$V$\cdot$希尔,他因在肌肉产热方面的研究而获得了诺贝尔奖。


里奇听从指令,用尽全力向前蹬,而比格兰则用力蹬着脚蹬,阻止他继续蹬。
想象一下,当观众看到这位娇小的女子轻而易举地阻止了这位身材高大的男人蹬得更快时,他们会有多么惊讶。
很快,他便大汗淋漓,气喘吁吁,而比格兰却几乎毫不费力。
甚至连伦敦市长后来也过来试驾。
“这套设备后来被命名为‘推拉你’,以纪念杜立特医生那只永远不知道自己要往哪个方向跑的双头怪兽,”布伦达$\cdot$比格兰-里奇后来写道。


魔术?小把戏?
并非如此,但它确实表明,肌肉消耗能量和产生力量的方式并非显而易见。
肌肉做正功(例如,在向前蹬踏时充当“马达”)时,它们消耗的能量和产生的热量,比做负功(例如,在向前蹬踏时充当“刹车”)时要多。
一般来说,肌肉产生的力量和消耗的能量,很大程度上取决于它是缩短还是伸长。
希尔实验的经验教训至今仍在日常康复和阻力训练中得到应用。
正如我们将看到的,奇迹发生在分子层面。


本章和下一章将深入探究肌肉内部,探索其结构与功能之间的关系。
肌肉是神奇的生物马达,能够在瞬间悄无声息地产生数千牛顿的力量。
这些力量如此巨大,以至于你小腿的肌肉就能举起一辆小型汽车的尾部。
这些巨大的力量是由数万亿个纳米级分子马达共同作用产生的,这些马达将我们摄入食物中的化学能转化为机械能,使我们能够活动。
这一非凡的功能源于骨骼肌特化的细胞机制和高度组织化的层级结构(图~\ref{fig:4_2})。


\begin{figure}[!htb]
	\centering
	\includegraphics[width=1.0\linewidth]{chap4/4_2}
	\caption{肌肉的多尺度结构。
		骨骼肌具有层次结构,其中有被称为肌球蛋白的纳米级分子马达,每个肌球蛋白仅产生几皮牛顿的力,排列成肌节、肌原纤维、纤维、肌束和整个肌肉,在强力肌肉收缩期间可产生数千牛顿的力。 \label{fig:4_2}}
\end{figure}


接下来两章的组织将大致遵循肌肉的层级结构。
我们将首先在分子层面研究力量的产生过程。
接下来,我们将了解分子马达是如何被包裹在被称为肌节的亚细胞结构中的(活体人体肌节的第一张图像就是在我的手臂上拍摄的,并在本章的开篇图中展示)。
进一步深入,我们将了解单个肌肉细胞如何被神经系统激活,并了解“快肌纤维”和“慢肌纤维”之间的区别。
在第~\ref{chap:chap5}~章中,我们将从宏观层面研究肌肉在体内的排列方式,以及它们如何与肌腱(将肌肉力量传递到骨骼结构)相互作用。


\section{肌肉结构}

从最基本的层面来说,肌肉通过两种细长蛋白质(肌动蛋白和肌球蛋白)的相互作用产生力量。
20 世纪 50 年代初,休$\cdot$赫胥黎通过 X 射线显微镜发现,这些蛋白质平行排列,纤维交织,它们之间的连接被他称为“横桥”。
安德鲁$\cdot$赫胥黎(与休无亲属)同时用不同的方法发现了横桥。
休$\cdot$赫胥黎和安德鲁$\cdot$赫胥黎都怀疑横桥是产生力量的机制,并于 1954 年提出了一个关于这些分子马达如何工作的模型,我们将在下文中解释。
他们的模型一直是解释力量产生的基本范式,尽管随着更多实验数据的收集,该模型变得更加详细。


从尺寸上看,肌纤维排列成束,称为肌束,它们与肌肉纤维一样,长度可达数十厘米。
肌束的横截面积约为 1 毫米。
除了肌纤维外,肌束还包含称为细胞外基质的结缔组织,其中包括胶原蛋白、神经纤维和血管。
在健康的肌肉中,肌纤维紧密排列;
然而,在患病的肌肉中,肌纤维的横截面积可能较小,并被更多的细胞外基质和脂肪隔开。


肌束被更多结缔组织包围,并聚集在一起形成肌肉。
另一层结缔组织鞘被称为筋膜,包裹着肌肉并将其与其他肌肉隔开。
终止于每个肌束末端的肌纤维可以直接附着在骨骼上,但通常它们会连接到肌腱,肌腱随后附着在骨骼上。
肌腱插入肌肉的部分称为腱膜;
肌腱在肌肉外部的部分通常称为游离肌腱。
正如我们将在第~\ref{chap:chap5}~章中看到的,肌腱不仅通过向骨骼传递肌肉力量发挥着重要作用,还在伸展和回缩时储存和释放能量。


\section{横桥循环}

当神经系统激活肌肉时,肌肉会通过数万亿个肌动蛋白和肌球蛋白的协同作用产生力量,这个过程被称为“横桥循环”(图~\ref{fig:4_3})。
这种机制可以粗略地描述为一个分子大小的棘轮。


\begin{figure}[!htb]
	\centering
	\includegraphics[width=1.0\linewidth]{chap4/4_3}
	\caption{横桥循环描述了肌动蛋白和肌球蛋白相互作用产生力量和运动的过程。
		A 代表腺苷;P 代表磷酸。 \label{fig:4_3}}
\end{figure}

肌球蛋白分子分为 3 个区域,分别称为头部、颈部和尾部。
肌球蛋白头部的独特结构使其能够牢固地结合到肌动蛋白丝上的特定位置(图~\ref{fig:4_3}~第一帧)。
当肌肉需要产生力量时,肌球蛋白头部会接收一种名为三磷酸腺苷 (ATP) 的“燃料”分子。
这刺激肌球蛋白头部从肌动蛋白上分离,向前旋转,并附着到肌动蛋白上的下一个结合位点(图~\ref{fig:4_3}~第二帧和第三帧)。
接下来,肌球蛋白头部围绕颈部区域旋转,这一运动被称为动力冲程 (power stroke),产生几皮牛顿的力,并使肌动蛋白丝彼此滑动约 10 纳米。


当然,不消耗能量就不可能完成机械功。
能量来自于一个化学反应,该反应将一个磷酸离子从三磷酸腺苷中分离出来(图~\ref{fig:4_3}~中的图4),留下二磷酸腺苷(ADP)。
最终,二磷酸腺苷被释放,肌球蛋白恢复到其原始状态,但现在已从其原始位置移位。
当肌球蛋白以这种方式循环时,细丝被拉向肌节中部,从而在化学能转化为机械能的过程中产生张力。


虽然这种机制听起来可能很复杂,但它却是一个优雅而经济的解决方案,解决了如何在分子层面上以可预测的方向产生力的问题。
就像汽车发动机一样,我们的肌肉将化学能转化为机械能,但噪音更小,产生的有毒废气也更少。



\section{肌节结构}

再往上一层,我们来看看肌节。
肌节大致呈圆柱形,长度根据肌肉长度在 1 微米到 5 微米之间变化。
肌节由许多相互交织的“粗”肌丝和“细”肌丝组成,随着肌节长度的变化,这些肌丝会相互滑动。


数百条肌球蛋白尾部捆绑在一起,形成每根粗肌丝,形成棒状结构,肌球蛋白头部以规则的间隔向外放射状延伸(图~\ref{fig:4_4})。
与粗肌丝平行的是细肌丝,它们由 3 种蛋白质组成:
肌动蛋白、原肌球蛋白和肌钙蛋白。
我们已经描述了肌动蛋白,它为肌球蛋白头部提供结合位点。
这些结合位点沿着细肌丝以规则的间隔分布。
原肌球蛋白和肌钙蛋白仅在肌肉被神经系统激活且存在钙离子时才会暴露这些结合位点,从而帮助调节力量的产生。
另一种值得关注的蛋白质(肌联蛋白),将每根粗肌丝附着在肌节的末端(我们称之为 Z 线或 Z 盘)。
正如我们稍后将看到的,肌联蛋白在被动力的产生中起着重要作用,这是一个独立于横桥循环的过程,也是赫胥黎最初的模型所忽略的现象。

\begin{figure}[!htb]
	\centering
	\includegraphics[width=1.0\linewidth]{chap4/4_4}
	\caption{肌球蛋白示意图(顶部)、粗丝和细丝的相互作用示意图(中间)以及肌原纤维横截面,显示粗丝和细丝以高度有序的三维模式排列(底部)。 \label{fig:4_4}}
\end{figure}

肌节在骨骼肌内以规则的模式串联和平行排列,在显微镜下观察骨骼肌组织时,会形成条纹,即明暗交替的带状结构。
您可以在图~\ref{fig:4_5}~中看到这些带状结构,它们分别被称为 I 带、A 带、Z 盘和 M 盘。
严格来说,Z 盘和 M 盘确实是圆盘,但它们通常被称为 Z 线和 M 线,因为它们在二维空间中看起来是这样的。
A 带仅出现在含有肌球蛋白的区域,而 I 带则出现在其他区域。细肌动蛋白丝的一端锚定在 Z 盘上。
粗肌球蛋白丝的一端附着在 M 盘的结构上,另一端通过肌联蛋白分子固定在 Z 盘上。
骨骼肌有时被称为“横纹肌”,与“平滑肌”相对,例如控制血管口径的肌肉,它们没有组织成肌节,也没有呈现条纹图案。

\begin{figure}[!htb]
	\centering
	\includegraphics[width=0.5\linewidth]{chap4/4_5}
	\caption{肌原纤维示意图(上)展示了其高度组织化的微观结构。
		肌肉显微镜图像(中)和肌节示意图(下)分别标记了I带、A带、M线和Z线。
		肌节的末端由Z线定义。 \label{fig:4_5}}
\end{figure}


\section{力-长度关系}

肌节所能产生的最大力量会随着其长度的变化而变化。
长度和张力之间的关系可以用滑动丝理论来解释,该理论由两个研究小组于 1954 年独立提出。
Andrew Huxley 和 Rolf Niedergerke(在单根肌纤维中)以及 Hugh Huxley 和 Jean Hanson(在分离的肌原纤维中)证明,在主动肌肉收缩过程中,肌小节带不会变窄,这推翻了粗肌丝缩短的流行理论。
他们给出的解释是,随着肌节长度的变化,粗肌丝和细肌丝会相互滑动。
因此,肌节的长度会影响肌动蛋白和肌球蛋白之间“重叠”的量,或者说靠近细肌丝上结合位点的肌球蛋白头部的数量。
随着肌球蛋白头部和肌动蛋白结合位点之间横桥数量的增加,激活的肌节内的张力也会增加。


如图~\ref{fig:4_6}~所示,激活的肌肉中横桥循环所能产生的力随肌节长度而变化。
该主动力-长度曲线通常被描述为具有三个区域:
上升区,其中力随肌节长度的增加而增加;
平台区,其中力保持在最大值;
下降区,其中力随肌节长度的增加而减小。
平台区涵盖一系列肌节长度,称为“最佳”范围,其中肌球蛋白头部和肌动蛋白结合位点之间的相互作用数量达到最大值。
最佳肌节长度因脊椎动物而异,但通常在 2.2 至 2.7 μm 之间,在人类骨骼肌中约为 2.7 μm。
当肌节长度超过最佳范围时,肌动蛋白和肌球蛋白的重叠量会减少,横桥循环所能产生的力量也会减少。
当肌节长度略短于最佳长度时,源自肌节两端的细肌丝开始在肌节中部重叠并相互干扰,导致所谓的“浅上升区”的力量减弱。
当肌节长度更短时(在“陡峭上升区”),粗肌丝会与Z盘碰撞并变形,产生阻碍横桥循环作用的力量。
由于肌肉由串联和并联排列的肌节组成,我们可以观察到肌肉主动收缩过程中力量与长度之间的类似关系。

\begin{figure}[!htb]
	\centering
	\includegraphics[width=0.5\linewidth]{chap4/4_6}
	\caption{肌节产生的主动力与其长度有关。粗肌丝和细肌丝相互滑过时,力会发生变化。
		在肌节的上升支,力随长度增加而增大,达到平台期;
		在肌节的下降支,力随长度增加而减小。
		当横桥数量达到最大时,力的产生达到峰值\cite{gordon1966variation}。 \label{fig:4_6}}
\end{figure}





\chapter{肌肉结构和动力学} \label{chap:chap5}


如果你想理解功能,那就研究结构。

\begin{flushright}
	——弗朗西斯$\cdot$克里克
\end{flushright}


\begin{figure}[!htb]
	\centering
	\includegraphics[width=1.0\linewidth]{chap4/4_0}
	% 加星号(*)表示不加编号
	\caption*{ \label{fig:5_0}}
\end{figure}


为了理解人类和动物的运动,研究人员开展了各种各样的实验。
生物力学家通过测量数千人的关节运动、地面反作用力和肌电信号来研究全身运动。
生理学家研究了单个肌肉,以表征肌肉激活和力量产生的动态。
肌肉驱动模拟使我们能够将这两个领域联系起来,将全身运动的生物力学测量与针对单个肌肉进行的实验相结合。
我们将在第~\ref{chap:chap10}~至~\ref{chap:chap12}~章中看到,肌肉驱动模拟可以深入了解肌肉在产生运动中的作用,并提供一些在人体运动时几乎无法测量的重要量的估计值,例如肌肉产生的力量和它消耗的能量。


肌肉动力学建模对于创建肌肉驱动的运动模拟至关重要。
然而,一刀切的模型并不适用,因为每块肌肉都有其独特的结构来适应其独特的功能。
例如,一些肌肉负责手指的精细运动控制,而另一些肌肉则在运动过程中支撑身体重量(图~\ref{fig:5_1})。
所有骨骼肌都具有肌节的层级排列,但肌肉在几个重要方面存在差异。
这些差异包括它们的大小和结构,以及肌肉纤维的几何排列。
因此,肌肉的计算模型必须捕捉所有肌肉共同的肌肉力量产生特征,同时还要能够表征每块肌肉的独特特征。


\begin{figure}[!htb]
	\centering
	\includegraphics[width=1.0\linewidth]{chap5/5_1}
	\caption{全身肌肉的结构和功能各不相同。
		浅屈指肌(最左侧)通过四条肌腱控制手指屈曲;
		宽阔的臀中肌和纤细的股薄肌产生髋关节外展和内收力矩;
		腓肠肌(最右侧)通过较长的跟腱止于跟骨。 \label{fig:5_1}}
\end{figure}


在本章中,我们将了解如何创建一个通用的肌肉力量产生模型,以及如何对其进行定制以代表身体中的几乎任何肌肉。
我们将要描述的肌肉模型属于以 A. V. Hill 命名的一类模型,除了我们在第~\ref{chap:chap4}~章中看到的“推我拉你”实验之外,他还进行了许多肌肉的基础研究。
我的博士生导师 Felix Zajac 改进了 Hill 型模型,并将其带入了现代计算机模拟时代\cite{zajac1989muscle}。
具体来说,Zajac 开发了一个仅包含四条通用曲线和五个肌肉特定参数的模型,所有这些参数都可以从实验数据中得出并用于调整模型。
Zajac 模型的简单性对于涉及数十块肌肉的动态模拟至关重要,但它足够详细,可以表示不同大小、强度和结构的肌肉的动态。


图~\ref{fig:4_18}~总结了所有肌肉共同的肌肉力量产生特征。
这些特征包括 3 条曲线,描述肌肉长度与其产生的力之间的非线性关系:主动力-长度曲线、被动力-长度曲线和力-速度曲线。
由于肌肉通过肌腱附着在骨骼上,我们还必须考虑这种结缔组织的特性,我们用肌腱的力-长度曲线来描述它。
我们使用 5 个参数缩放这些通用曲线以表示特定的肌肉:
(1)最佳肌纤维长度 $l_o^M$;
(2)最佳纤维长度下的肌纤维羽状角 $\phi_o$ ;
(3)最大等长肌肉力量,$F_o^M$;
(4)最大肌肉收缩速度 $v_\text{max}^M$;
和(5)肌腱松弛长度,$l_s^T$。
本章首先介绍这 5 个特定于肌肉的参数。
我们将了解每个参数如何影响肌肉力量,并将其纳入肌肉-肌腱动力学模型中。


\section{最佳肌纤维长度$l_o^M$}

正如我们在第~\ref{chap:chap4}~章中看到的,肌节能够产生的主动力取决于其长度(图~\ref{fig:4_6})。
肌节能够产生最大等长收缩力的长度称为其最佳长度。
由于肌纤维由多个($n$)个首尾相连的肌节组成,因此肌纤维也存在一个最佳长度($l_o^M$),当其每个组成肌节都达到其最佳长度($l_o^S$)时,肌纤维便会达到该最佳长度:
%
\begin{equation}
	l_o^M = n l_o^S \label{eq:5_1}
\end{equation}

公式~\ref{eq:5_1}~假设肌纤维上串联的所有肌节长度相同。
肌肉在运动过程中会伸长和缩短,这会影响肌节粗肌丝和细肌丝相互滑动时产生的主动力。
最佳长度较长的肌纤维(即串联肌节较多)具有更宽的主动力-长度曲线,并且可以在更宽的长度范围内产生其最大主动力的很大一部分(图~\ref{fig:5_2},顶部)。
增加肌纤维的最佳长度也会增加其最大缩短速度():

\begin{figure}[!htb]
	\centering
	\includegraphics[width=1.0\linewidth]{chap5/5_2}
	\caption{最佳肌纤维长度较长的肌肉,其主动力-长度曲线(上图)更宽,最大缩短速度也更高(下图)。
		请注意,长肌纤维(蓝色)的示意图中,肌节数量是短肌纤维(橙色)的两倍,因此长肌纤维在给定时间内可以缩短 2 倍的距离(右下图)。 \label{fig:5_2}}
\end{figure}

\begin{equation}
	v_{\text{max}}^M = n v_{\text{max}}^S \label{eq:5_2}
\end{equation}
%
其中,$v_\text{max}^S$表示肌节的最大缩短速度。
因此,随着肌肉最佳纤维长度的增加,力-速度曲线也会变宽(图~\ref{fig:5_2},底部)。


生物肌肉由长度不同的肌束构成,肌束本身包含长度也不同的纤维,这些纤维甚至可能终止于肌束内。
然而,在我们的模型中,我们假设肌肉中的所有纤维长度相同——许多(但并非所有)肌肉-肌腱动力学模型都做出了这一假设。
我们进一步假设所有纤维都是直的、平行的且共面的。
因此,为了表征肌肉的力-长度和力-速度特性,我们只是放大了肌纤维的相应特性,而这些特性仅仅是肌节相同特性的放大版本。



\section{最佳纤维长度下的肌纤维羽状角$\phi_o$}

肌肉通常通过肌腱附着于骨骼。
在平行纤维肌腱中,例如缝匠肌,其纤维沿着肌腱方向排列(图~\ref{fig:5_3})。
在大多数其他肌肉中,例如股直肌,其纤维与肌腱呈锐角排列;我们称这些肌肉为羽状肌。
“羽状肌”一词源于拉丁语,意为“羽毛状”,而羽状肌的结构确实让人联想到鸟类的羽毛。

\begin{figure}[!htb]
	\centering
	\includegraphics[width=0.8\linewidth]{chap5/5_3}
	\caption{具有不同结构的肌肉的例子:
		平行纤维肌肉、单羽状肌肉、双羽状肌肉和多羽状肌肉。 \label{fig:5_3}}
\end{figure}

如果所有肌纤维都附着在肌腱的一侧,我们称该肌肉为单羽状肌;如果肌纤维附着在肌腱的两侧,则称该肌肉为双羽状肌。
在多羽状肌中,肌腱分支和肌纤维结构可能很复杂。
因此,我们假设给定肌肉中的所有纤维都以相同的角度(称为羽状角 ($\phi$))附着于肌腱,并采用图~\ref{fig:5_4}~所示的肌肉-肌腱几何模型。
由此,我们得到肌纤维中的力 ($F^M$) 和肌腱中的力 ($F^T$) 之间的以下关系:

\begin{figure}[!htb]
	\centering
	\includegraphics[width=0.75\linewidth]{chap5/5_4}
	\caption{肌纤维和肌腱的简化几何表示。
		肌纤维被假设为直的、平行的、共面的、等长的,并以相同的羽状角 ($\phi$) 附着于肌腱。
			当肌纤维缩短或伸长,羽状角增大或减小时,平行四边形的高度 $h$(以及面积)保持不变\cite{zajac1989muscle}。 \label{fig:5_4}}
\end{figure}

\begin{equation}
	F^T = F^M cos(\phi) 
	\label{eq:5_3}
\end{equation}

现在,参考图~\ref{fig:5_4},我们可以解释生物肌肉如何在纤维长度发生变化的情况下保持体积恒定。
随着图中纤维的缩短,想象一下它们以这样的方式缩短,使得图~\ref{fig:5_4}中的平行四边形保持相同的高度 $h$。
平行四边形的顶部将保持不变,但底部将被向右拉,平行四边形将变得更接近矩形。
然而,只要高度保持不变,面积也将保持不变,这符合平行四边形面积等于其底边和高乘积的几何规则。
我们以二维方式绘制了该图,但三维运动类似,并确保肌肉的体积不变。
简而言之,羽状肌不是通过膨胀来维持体积,而是通过剪切来维持体积。


在上述过程中,羽状角不断增大,从纤维传递到肌腱的力不断减小,直到纤维与肌腱垂直,图中的肌肉呈矩形(即$\phi$ = 90度)。
我们使用参数$\phi_o$表示肌纤维达到最佳长度时的羽状角(即$l^M = l_o^M$)。
我们所描述的固定高度近似法可能会对收缩时明显隆起的肌肉引入误差,但它为研究肌肉结构的功能含义提供了一个简单的几何模型。


除了公式~\ref{eq:5_3}~中表达的关系外,羽状肌在决定肌肉的产力能力方面也起着至关重要的作用。
一般来说,羽状肌角度越大,在给定体积内能够容纳的肌肉纤维越多。
想象一下在矩形房间铺设硬木地板的类似情况:可以使用相对较少的长木板来延伸房间的长度,或者使用大量较短的木板以对角线方向铺设。
同样,与相同体积的平行纤维肌肉相比,羽状肌的纤维更短,因此主动力-长度曲线和力-速度曲线更窄(图~\ref{fig:5_2})。
当然,羽状肌也包含更多的纤维,其后果将在下一节探讨。



\section{最大等长肌肉力量$F_o^M$}

我们 5 个肌肉参数系列中的第三个参数相对容易理解,但测量起来却不那么容易。
这就是最大等长肌肉力。
它被定义为肌肉在最大程度激活并保持最佳纤维长度时产生的力量。


对于活体人体来说,最大等长肌力很难测量,因为我们无法将一块肌肉与其他肌肉分离,并只对该肌肉施加阻力。
因此,我们使用一个称为生理横截面积(PCSA;图~\ref{fig:5_5})的指标。
这是肌肉垂直于纤维方向的横截面积。
需要注意的是,在羽状肌中,该横截面积与肌肉的纵轴倾斜。
最大等长肌力可以通过以下方式估算:

\begin{figure}[!htb]
	\centering
	\includegraphics[width=1.0\linewidth]{chap5/5_5}
	\caption{左图所示的肌肉体积相同,但PCSA(主成分分析面积)、最佳纤维长度和羽状角不同。
		羽状肌越多,产生的主动力越大,但纤维越短;
		因此,其主动力-长度曲线和动力-速度曲线更高,但更窄\cite{lieber2002skeletal}。 \label{fig:5_5}}
\end{figure}

\begin{equation}
	F_o^M = \text{PCSA} \sigma_o^M
	\label{eq:5_4}
\end{equation}
%
其中 $\sigma_o^M$ 是肌肉的比张力(也称为峰值等长应力),即单位面积可产生的最大肌肉力。
在构建健康肌肉模型时,该参数的典型值为 $\sigma_o^M = 0.3 \; \text{MPa}$ 。


在上一节中,我们注意到,羽状肌的纤维比相同体积的平行纤维肌更短,但数量也更多。
因此,尽管羽状肌的力-长度和力-速度曲线较窄,但这些曲线也会更高,因为羽状肌的PCSA(以及因此产生的最大等长力)更大(图~\ref{fig:5_5})。


肌肉纤维的长度和收缩速度不仅仅受其几何形状的影响,我们将在接下来的两节中看到这一点。
尽管如此,我们已经可以观察到肌肉的结构如何影响其所能执行的功能范围:
在其他条件相同的情况下,羽状肌能够比相同体积的平行纤维肌产生更大的力量,但其长度范围较小,收缩速度也较低。



\chapter{肌肉骨骼几何学} \label{chap:chap6}



\chapter{运动量化} \label{chap:chap7}





\chapter{逆动力学} \label{chap:chap8}












\chapter{肌肉力量优化} \label{chap:chap9}

人类遇到的每一个问题总有一个众所周知的解决方案——简洁、合理,但又错误。
\begin{flushright}
	——H. L. 门肯
\end{flushright}


\begin{figure}[!htb]
	\centering
	\includegraphics[width=1.0\linewidth]{chap9/9_0}
	% 加星号(*)表示不加编号
	\caption*{ \label{fig:9_0}}
\end{figure}


我一生中发现的最伟大的生物力学发现之一并非源于实验室,也未借助任何特殊设备,仅仅依靠人体不可思议的适应性。
1968年,一位名叫迪克$\cdot$福斯贝里的21岁土木工程系学生公布了他的发现,并在奥运会上以向后跳过跳高横杆的方式震惊了体育界。
一位体育记者写道:“福斯贝里跳过横杆就像一个人被人从30层楼的窗户推下去一样。”
然而,正是这种看似笨拙的跳高方式,让他创造了 2.24 米的奥运纪录,并最终获得金牌。


福斯贝里实际上是在高中时出于无奈才开始使用他的标志性技术。
他没能掌握当时的标准技术——“西部翻滚”,即跳高运动员面朝下越过横杆,仿佛用手臂和腿环抱横杆。
他还尝试了更古老的“剪刀”技术,即跳高运动员以近乎坐姿的姿势越过横杆,同时双腿像剪刀一样上下摆动。
但这种技术并不理想,因为跳高运动员必须将重心推到远高于横杆的高度。


在背越式跳高中,跳高运动员先向横杆一端跑去,然后向内弯曲身体,向中间跃过横杆,最后在最后一刻扭转身体,向后跃过横杆(图~\ref{fig:9_1})。
身体每次只滑过一个部位:
首先是头部和肩部,然后是躯干、臀部、膝盖,最后是双脚。
背越式跳高的一个优点是身体重心无需越过横杆。
在最高高度,臀部位于横杆上方时,背部拱起,头部和腿部悬在横杆下方,从而降低了重心。


\begin{figure}[!htb]
	\centering
	\includegraphics[width=1.0\linewidth]{chap9/9_1}
	\caption{“背越式跳高”技术。
		玛雅扬$\cdot$弗曼-沙哈夫的照片,摄影:埃夫伦$\cdot$卡林巴卡克。 \label{fig:9_1}}
\end{figure}


尽管福斯贝里在发明背越式跳高时并未接受过任何正规的生物力学训练,但我们现在已经赶上了他开创性的天才。
如今,教练们使用高速录像来指导跳高运动员脚部着地的位置、蹬地前蹲到多低等等。
他们想到了福斯贝里从未想过的改进。
起跳时抬起手臂可以增加垂直动能。
臀部越过横杆后收下巴有助于迫使臀部下沉,并根据牛顿第三运动定律,将双脚抬起并越过横杆。


自 1980 年以来,所有跳高世界纪录都是用背越式跳高创造的,其他所有技术都已从世界舞台上消失。
现在争论的焦点是哪种背越式跳高最好:
“速度背越式”还是“力量背越式”?


对我来说,福斯贝里跳马的故事教会了我几个宝贵的教训。
首先,我们永远不应该想当然地认为流行的做事方式就是唯一或最佳方式。
但一个不那么哲学、更偏向生物力学的信息是,我们的身体拥有关节骨骼和数百块肌肉,赋予我们完成任务的无限可能,无论是像端咖啡这样看似简单的动作,还是像跳 2 米高这样复杂的动作。
每一个动作都需要我们的大脑协调肌肉活动。
下巴的一个简单的动作就能对我们的双脚产生影响。


为了完成任何动作,我们的身体都会解决一个优化问题,试图用最小的努力获得最大的效果。
但有一个问题,无论是数学模型还是现实世界的运动员,都可能遇到难题:局部最优并不总是全局最优。
在背越式跳高之前的运动员们都认为他们找到了最佳技术。
他们错了。
但他们无法仅仅通过对西方的滚翻技术进行小幅(“局部”)调整来提高标准。
他们需要一位愿意进行大规模(“全局”)调整的运动员工程师,来改变我们对跳高应该是什么样子的理解。


在本章中,我们将探讨 2 种使用数值优化计算肌肉力量的广泛场景。
第一种场景出现在我们希望\textit{估算产生可观测运动的肌肉力}时。
由于人体肌肉骨骼系统存在冗余,因此通常存在许多可能的解,我们将应用优化方法,找到符合某些标准的最佳肌肉力量集。
我们将这个问题称为肌肉冗余问题(也称为肌肉力量共享问题)。
第二种场景出现在研究肌肉协调性以\textit{优化特定任务的表现}时(例如跳高)。
在这种情况下,我们需要计算肌肉力以及完成任务的身体节段运动。
如果为优化器提供跳高比赛规则和足够逼真的肌肉骨骼系统模型,它就能发现背越式跳高(以及其他我们可能从未见过的技术)。
需要注意的是,使用优化方法估算肌肉力量不仅仅是为了计算上的便利:神经系统本身就是一种优化器。
让我们来深入了解一下细节。


\section{生物和数值优化器}

人们有意识地优化他们的日常生活。
我们会决定如何分配时间、金钱、注意力和其他有限的资源,以最大限度地平衡短期和长期的满足感,并在遵守法律和其他义务等约束的同时做到这一点。
令人惊讶的是,人类也会在潜意识中优化。
回想一下第二章,我们会自然地调整步行速度、步频和其他步态变量,以使我们的交通成本接近最低——当然,除非我们赶时间,在这种情况下我们会优先考虑速度而不是代谢成本。
在学习新任务和适应新情况时,我们也会在优化过程中,根据具体情况最大化准确性、效率、舒适度和其他性能指标的某种组合。
实验表明,我们会维持一个关于身体的“内部模型”,即潜意识地理解我们刺激肌肉的程度与由此产生的身体部位运动之间的关系。
当我们置身于新环境,例如国际空间站的微重力环境或机器人装置施加的人工力场时,我们最初会显得笨拙且效率低下,因为我们在规划和预测自身动作时使用的是一个不再精准的内部模型。
在探索新环境的过程中,我们会利用视觉和其他感官反馈来调整内部模型,提高协调性。
随着时间的推移,我们会适应新环境,并通过重新学习肌肉兴奋与运动表现之间的关系,重新获得熟练的技能。


数值优化算法使用类似的探索性或“猜测-检验”策略来寻找欠定问题的解。
我们将这些猜测称为候选解,每个候选解都为所有未知数或设计变量(例如肌肉冗余问题中的肌肉激励)提供了一个数值。
每个候选解的适用性通过评估目标函数或成本函数来确定,成本函数是量化特定设计变量值集合的可取性的表达式。
能够提供最佳目标函数值(视情况而定,最小值或最大值)的候选解被称为最优解,或者简称为优化问题的解(请注意,我们可以将我们的任务定义为始终寻求最小化目标函数;为了最大化像跳跃高度这样的量,我们只需最小化其负值即可)。
正如我们将看到的,目标函数的选择会对解以及计算它所需的工作量产生深远的影响。


最优解也会受到约束条件的影响。
在许多问题中,我们要求候选解满足某些表达式(等式或不等式),才能使其成为可行解或可接受的解。
例如,在肌肉冗余问题中,我们要求所有肌肉力均为拉伸力,并且它们产生的力矩之和等于所需的净关节力矩(例如,使用逆动力学算法计算)。
由于约束条件只能减少可行解的数量(即减小解空间的大小),因此,受约束优化问题的解的目标函数值并不比优化问题不受约束时更好。
添加约束会减少您的选择。
如果约束条件过多,解空间可能为空(问题可能没有可行解)。
根据研究问题的不同,可以将一个或多个严格约束转换为软约束,这些软约束以“惩罚”的形式添加到目标函数中。
惩罚项通常通过加权因子进行缩放,以便根据其相对重要性调整它们对目标函数的贡献。
通用优化问题可以表述如下:
%
\begin{proposition}[优化问题 1:找到产生所需关节力矩的踝跖屈肌力量] \label{pro:optim_1}
	
	\begin{equation}
		\begin{matrix}
			\text{最小化} & $ J(\underline{x}) $  & \text{调整设计变量 $ \underline{x} $ 以最小化目标函数 $ J(\underline{x}) $} \\
			% 
			$\text{受约束}$ & g_i(\underline{x}) \leq 0, i=1,...,n^i  & \text{同时满足 $ n $ 个不等式约束,} \\
			%
			& h_j (\underline{x}) = 0, j=1,...,n^j  &  \text{满足 $ n $ 个等式约束,} \\
			& \underline{x}^\text{lower} \leq \underline{x} \leq \underline{x}^\text{upper} & \text{并尊重设计变量的界限。}
		\end{matrix} \nonumber
	\end{equation}
	
\end{proposition}


我们潜意识地运用同样的原理来协调肌肉。
想象一下拿起桌上一支笔的动作。
你知道手所需的最终位置和方向,但你可以自由地(在一定范围内)选择手臂的最终姿势以及达到该姿势的路径。
正式地说,这个问题的每个候选解可能提出一组不同的肌肉激励,这些激励被定义为时间的参数化函数;这些参数将成为设计变量。
在所有候选解中,可行解是那些描述实现所需最终手部姿势的肌肉激励模式的解。
其他约束条件可能描述其他不可协商的要求,例如避开障碍物。
你大概会寻求在一定程度上最小化行程时间,但目标函数也可能包含一个有利于减少肌肉力量的项,用于完成这项非紧急任务。
解决此类问题的一种简单方法是评估所有可行解的目标函数,然后选择最佳候选解。
然而,正如您所料,所谓的“穷举搜索”(即考虑所有可能性)对于除最简单问题之外的所有问题都是低效且不切实际的。
数值优化器会尝试通过评估其认为必要的候选解数量来找到最有利的可行解(或至少是足以解决当前问题的可行解)。
许多优化器的区别仅在于它们如何根据已探索的候选解的适用性来选择新的候选解进行评估。


现在,我们将转向讨论解决肌肉冗余问题的策略。
为简单起见,我们将在某一时刻寻求解,假设所需的净关节矩已知,且对象静止不动。
我们有时将此类问题称为“静态”优化问题,因为其中忽略了一些与时间相关的因素。
在下一节中,我们将构建一个优化问题陈述,并通过观察求解一个简单的肌肉冗余问题,以建立对解空间的直觉。
在接下来的章节中,我们将讨论一些能够解决现实世界中遇到的非平凡优化问题的数值算法。



\section{通过检查解决静态优化问题}

我们首先关注图~\ref{fig:9_2}~所示的模型。假设我们希望产生 100 N$ \cdot $m 的踝关节跖屈净力矩。
应该激励哪些肌肉来产生这个力矩呢?有很多可行的方案。
例如,也许整个期望力矩都应该由比目鱼肌产生,或者最好是模型中的 3 个跖屈肌各自贡献一部分力矩。
背屈肌也可以产生力量,也可以由其中 1 块或 2 块背屈肌产生。
因此,该问题是欠定的:方程的数量少于未知数,我们需要更多信息来解决这个问题。
正式表述为,我们希望找到由每块肌肉 $ i $(我们的设计变量)产生的力 $ F_i $,使得踝关节跖屈总力矩为 100 N$ \cdot $m(等式约束),但前提是 $ F_i $ 为正,且对于任意 $ i = 1, ..., 5 $(界限)。


\begin{figure}[!htb]
	\centering
	\includegraphics[width=0.7\linewidth]{chap9/9_2}
	\caption{小腿和足部的肌肉骨骼模型,包含关键的跖屈肌和背屈肌。
		测量的踝关节力矩可能由多种肌肉力量组合产生。
		括号中的值是瞬时最大力(假设速度为零且肌腱刚性)以及所示姿势对应的踝关节力臂。 \label{fig:9_2}}
\end{figure}


解决欠定问题的一种方法是修改问题,使方程的数量与未知数的数量相同。
在我们的示例中,为简单起见,我们假设背屈肌处于非活动状态,产生的力为 0。
% 腓肠肌:gastrocnemius 
% 比目鱼肌:soleus
% 胫骨后肌:tibialis posterior
这一假设将未知数的数量从 5 个减少到 3 个:腓肠肌(我们称之为 $ F_\text{GAS} $)、比目鱼肌($ F_\text{SOL} $)和胫骨后肌($ F_\text{TP} $)产生的力。
然而,我们仍然只有一个方程(指定期望净踝关节力矩的等式约束)。
我们可以假设每块跖屈肌都会产生相同的力,并引入 2 个附加方程,$ F_\text{SOL} = F_\text{GAS} $ 和 $ F_\text{TP} = F_\text{GAS} $。
这种策略可以得到一个可行的解,但通常不会产生生理上合理的结果。
相反,我们将定义一个目标函数 $ J(\underline{F}) $,它将解的有利性量化为肌肉力的函数,然后求解以下优化问题:
%
\begin{proposition}[优化问题 1:找到产生所需关节力矩的踝跖屈肌力量]

\begin{equation}
	\begin{aligned}
		\text{最小化} & J(\underline{F})  &  \text{目标函数中较小的值更受青睐} \\
		\text{受限于} & 0.039 F^\text{GAS} + 0.036 F^\text{SOL} + 0.008 F^\text{TP} & \text{肌肉必须产生所需的净踝关节力矩} \\
		& 0 \leq F^\text{GAS} \leq 4097 & \\
		& 0 \leq F^\text{SOL} \leq 6435 & \text{肌肉力量必须在生理范围内} \\
		& 0 \leq F^\text{TP} \leq 3052 &  \nonumber
	\end{aligned}
\end{equation}

\end{proposition}


例如,假设我们假设神经系统最小化每个时刻产生的总肌肉力,因此这就是瞬时肌肉力的总和:
%
\begin{equation}
	J ( \underline{F} ) 
		\triangleq
		F ^\text{GAS} + 
		F ^\text{SOL} + 
		F ^\text{TP} 
		= \sum_{i=1}^{3} F_i
	\label{eq:9_1}
\end{equation}

本例中的解是 $F^\text{GAS} = 2564$、$F^\text{SOL} = 0$ 和 $F^\text{TP} = 0$,因为腓肠肌具有最大的力臂,因此可以用最小的力量产生所需的踝关节跖屈力矩。
需要注意的是,在本模型中,腓肠肌能够产生完整的 100 N$\cdot$m 跖屈力矩(即 2564 N < 4097 N),实际上,当腓肠肌完全激活时,它可以在此姿势下产生约 160 N$\cdot$m 的力矩。
超过 160 N$\cdot$m 的踝关节力矩可以通过募集比目鱼肌(力臂第二大的肌肉)来产生,然后在比目鱼肌也达到最大力量后,再募集胫骨后肌(图~\ref{fig:9_3})。

\begin{figure}[!htb]
	\centering
	\includegraphics[width=0.75\linewidth]{chap9/9_3}
	\caption{图~\ref{fig:9_2}~中,假设肌肉协调策略最小化肌肉力量总和,则每块肌肉施加的力量会产生踝关节跖屈力矩。
		该目标函数(公式~\ref{eq:9_1})会导致不切实际的行为,即每块肌肉在下一块肌肉被调动之前就达到了其最大力量。 \label{fig:9_3}}
\end{figure}


虽然公式~\ref{eq:9_1}~中的目标函数可以很容易地求解优化问题,但请注意,当需要 100 N$\cdot$m 的跖屈力矩时,只会募集一块肌肉。
如果需要更大的力矩,则只有当第一块肌肉达到其最大产力能力时,我们才会募集第二块肌肉。
我们从实验中得知,肌肉实际上并非以这种逐步、非合作的方式募集。


不同的目标函数会导致不同的解决方案。
生物力学文献中提出了一些更切合实际的目标函数,但这些目标函数会导致无法通过检验解决的优化问题。
通常,我们必须使用计算机来解决生物力学中遇到的非线性、约束、高维问题。
在接下来的部分中,我们将讨论两大类优化算法:局部方法和全局方法。
然后,我们将继续解决一个更切合实际的肌肉力共享问题。



\section{解决静态优化问题的局部方法}

局部优化方法从一个初始猜测(候选解)开始,生成一系列新的猜测,每个猜测都与其前一个猜测相邻,并且通常具有更优的目标函数值。
当算法达到局部最优值时终止:该解在其紧邻域中被劣质解包围。
局部方法的一个典型示例是最速下降算法。
例如,如果我们求解一个二元最小化问题,目标函数 $J(x,y)$ 可以表示为一个曲面,其在点 $(x^*,y^*)$ 处高于 X-Y 平面的高度为 $J(x^*,y^*)$。
最速下降策略可以想象为将一颗弹珠放在目标函数曲面上,让它滚下山坡,直到落到盆底。
如图~\ref{fig:9_4}~所示,最终解取决于初始猜测,并且可能不是最佳解。
牛顿法与最速下降法类似,但它利用目标函数曲面的二阶导数(曲率)信息,更直接地逼近局部最小值——尽管计算这些额外信息的成本可能相当高。
内点法也很流行,它首先找到任意可行解(即可行域内部的解),然后在可行域内逐步寻找更优解。


\begin{figure}[!htb]
	\centering
	\includegraphics[width=1.0\linewidth]{chap9/9_4}
	\caption{两个变量目标函数的图形表示,其中曲面高度表示目标函数值 $J(x,y)$。
		最速下降算法从解空间中的某个点开始,沿着局部斜率最大的方向逐步下行,直到沿任何方向移动都能使 $J(x,y)$ 增加。
		最终解取决于初始猜测值,如此处所示的四条路径所示,并且可能并非全局最优。 \label{fig:9_4}}
\end{figure}


在图~\ref{fig:9_5}~中,我们展示了优化问题~\ref{pro:optim_1}(上文)的解决方案,该方案针对所有可能的踝关节跖屈力矩,同时最小化肌肉激活平方和,该平方和已被用作代谢能量消耗的替代品:


\begin{figure}[!htb]
	\centering
	\includegraphics[width=0.85\linewidth]{chap9/9_5}
	\caption{图~\ref{fig:9_2}~中,假设肌肉协调策略最小化肌肉激活平方和,则每块肌肉为产生所有可能的踝关节跖屈力矩而施加的力。
		(公式~\ref{eq:9_2})得出的行为与实验观察结果一致,即即使产生微小的关节力矩,也会调动多块肌肉。 \label{fig:9_5}}
\end{figure}

\begin{equation}
	J ( \underline{F} ) 
		\triangleq 
		\sum_{i=1}^{3}
			( \frac{F_i}{F_i^\text{max}} )^2
		= 
		\sum_{i=1}^{3} a_i^2
	\label{eq:9_2}
\end{equation}


使用公式~\ref{eq:9_2}~计算出的肌肉活动比使用公式~\ref{eq:9_1}~所示的更简单的目标函数更接近人体实验中观察到的情况。
特别需要注意的是,在图~\ref{fig:9_5}~中,即使只需要较小的踝关节跖屈力矩,所有 3 块肌肉都会被募集,这与\textit{肌电图}的实验测量结果(图~\ref{fig:9_6})在定性上相似。


\begin{figure}[!htb]
	\centering
	\includegraphics[width=1.0\linewidth]{chap9/9_6}
	\caption{行走时腓肠肌外侧肌(左)和比目鱼肌(右)的\textit{肌电图}信号。
		这两块肌肉都参与踝关节跖屈力矩的产生\cite{bartlett2008changing,arnold2013muscle}。 \label{fig:9_6}}
\end{figure}


局部方法寻求其邻域内的最优解;然而,这样的局部最小值可能有很多。
例如,如图~\ref{fig:9_4}~所示,梯度下降算法找到的解取决于初始猜测。
我们或许可以尝试用不同的初始猜测运行该算法几次,但这仍然可能错过更好的(甚至可能好得多的)解。
记住迪克$\cdot$福斯贝里(Dick Fosbury)的名言。
此,尽管局部方法可能很快,但全局方法可能会产生更好的解。



\section{解决静态优化问题的全局方法}

全局优化方法通过考虑先前猜测值附近以外的候选解,避免“卡”入局部极小值。
遗传算法就是这样一种算法,其中候选解的种群通过模拟自然过程从一个阶段(一代)进化到下一个阶段。
这种优化策略背后的理念是更详细地探索解空间中已知有利的区域,同时也探索解空间中可能包含更优解的未探索区域。


遗传算法往往难以随问题规模的扩大而扩展,但其他进化算法在实践中表现良好。
例如,图~\ref{fig:9_7}~所示的\textit{协方差矩阵自适应调整的进化策略}对于存在许多局部最小值的高维问题表现非常出色。
在每一代中,\textit{协方差矩阵自适应调整的进化策略}算法都会从一个分布中选择候选解,该分布的均值和协方差会根据前几代的偏好度进行更新。
随着时间的推移,该分布将逐渐趋近于一个解:均值将趋近于最优值,而分布则会收缩。
模拟退火算法采用了类似的思路;算法开始时会处于较高的“温度”,允许对解空间进行更具冒险精神的探索,然后逐渐“冷却”到更为保守的最速下降法。


\begin{figure}[!htb]
	\centering
	\includegraphics[width=1.0\linewidth]{chap9/9_7}
	\caption{\textit{协方差矩阵自适应调整的进化策略}等全局优化方法可以避免收敛到局部最小值。
		图中展示了\textit{协方差矩阵自适应调整的进化策略}算法的九代迭代过程,该算法最小化了图~\ref{fig:9_4}~所示的函数。
		每个青色点代表一个候选解;每个面板中的椭圆表示该代算法的种群分布。 \label{fig:9_7}}
\end{figure}


最佳优化器的选择取决于问题的性质。
例如,如果目标函数平滑且问题只有一个最小值,那么简单的局部搜索策略就足够了,因为它可以保证收敛到全局最小值。
另一方面,当存在许多局部最小值时,像\textit{协方差矩阵自适应调整的进化策略}这样的算法会更合适。
对于大型问题,使用可以并行运行的算法并定义标准来判断何时解足以解答特定的研究问题会更有利。



\section{行走和跑步时的肌肉力量}

有了这些优化方法,我们现在能够解决生物力学中最重要的问题之一:
确定负责产生行走和跑步等动作的肌肉力量。
这被称为肌肉冗余(或力量共享)问题。


当我们跑步时,在站立阶段的第一阶段,我们的肌肉会产生髋部伸展、膝部伸展和踝部跖屈的力矩。
假设我们希望计算在垂直地面反作用力达到峰值时产生的肌肉力(图~\ref{fig:9_8})。
使用图~\ref{fig:9_9}~中给出的下肢模型和数据,以及公式~\ref{eq:9_2}~中给出的目标函数来最小化肌肉激活平方和,我们可以将优化问题表达如下:


\begin{figure}[!htb]
	\centering
	\includegraphics[width=0.65\linewidth]{chap9/9_8}
	\caption{一名受试者以 5 米/秒的速度奔跑时在矢状面上产生的净关节力矩。
		垂直线表示垂直\textit{地面反作用力}峰值出现的时间\cite{hamner2013muscle}。 \label{fig:9_8}}
\end{figure}


\begin{figure}[!htb]
	\centering
	% 设置宽度为0.75会导致所有章节不显示正文
	% 可能是由于干扰“优化问题 2”框导致
	\includegraphics[width=0.7\linewidth]{chap9/9_9}
	\caption{一个简单的腿部肌肉骨骼模型可用于研究跑步站立阶段的肌肉协调性和关节负荷。
		参与产生矢状面运动的关键肌肉已被归类为 9 条代表性肌肉路径\cite{hamner2013muscle}。
		括号中的值是对应于所示姿势的瞬时最大力(假设速度为零且肌腱刚性)和力臂。 \label{fig:9_9}}
\end{figure}


\begin{proposition}[优化问题 2:找到跑步过程中垂直地面反作用力达到峰值时的肌肉激活情况]
	
	\begin{equation}
		\begin{aligned}
			\text{最小化} & J(\underline{a}) = \sum_{i=1}^{9} a_i ^2  \\
			\text{受限于} & \sum_{i=1}^{9} a_i ( r_i^\text{hip} F_i^\text{max} ) = -67.3 \\
			%
			& \sum_{i=1}^{9} a_i ( r_i^\text{knee} F_i^\text{max} ) = -139 \\
			& \sum_{i=1}^{9} a_i ( r_i^\text{ankle} F_i^\text{max} ) = -206 \\
			& 0 \leq a_i \leq 1 \; \text{for} \; i=1,...,9
			\nonumber
		\end{aligned}
	\end{equation}
	
\end{proposition}

该优化问题的解决方案是估计模型中每个肌肉在某一时间点产生的力量:

\begin{table}[htbp]
	\label{tab:9_1} \centering
	\begin{tabular}{cc} % l水平左居中,c水平居中
		\toprule
		肌肉或肌肉群 & 力, $F_i$牛  \\
		\midrule
		臀大肌 & 875  \\
		\midrule
		髂腰肌 & 0  \\
		\midrule
		腘绳肌 & 340  \\
		\midrule
		股直肌 & 0  \\
		\midrule
		股二头肌短头 & 0  \\
		\midrule
		股肌 & 4134  \\
		\midrule
		腓肠肌 & 1396  \\
		\midrule
		比目鱼肌 & 4167  \\
		\midrule
		胫骨前肌 & 0  \\
		\bottomrule
	\end{tabular}
\end{table}


通过在步态周期中均匀间隔的时刻重复此分析,我们可以估算出每块肌肉在行走和跑步过程中产生的力量。
图~\ref{fig:9_10}~和图~\ref{fig:9_11}~显示了行走状态下的肌肉力量及其产生的关节力矩;
图~\ref{fig:9_12}~和图~\ref{fig:9_13}~显示了跑步状态下的肌肉力量及其产生的关节力矩。
正如预期的那样,计算出的跑步状态下的肌肉力量高于行走状态下的肌肉力量。
还要注意的是,由于站立阶段较短,跑步状态下肌肉力量在步态周期中更早达到峰值,这也是我们预期的。


\begin{figure}[!htb]
	\centering
	\includegraphics[width=1.0\linewidth]{chap9/9_10}
	\caption{使用 OpenSim 中的静态优化计算了一名受试者(男性,67.1 公斤)以自由选择的速度(1.67 米/秒)行走时的肌肉力量\cite{dembia2017simulating}。 \label{fig:9_10}}
\end{figure}


\begin{figure}[!htb]
	\centering
	\includegraphics[width=0.65\linewidth]{chap9/9_11}
	\caption{如图~\ref{fig:9_10}~所示,以 1.67 m/s 的速度行走时,肌肉力量产生的矢状面关节力矩。 \label{fig:9_11}}
\end{figure}


\begin{figure}[!htb]
	\centering
	\includegraphics[width=1.0\linewidth]{chap9/9_12}
	\caption{一名受试者(男性,69.4 公斤)以 5 米/秒的速度奔跑时的肌肉力量,使用 OpenSim 中的静态优化计算得出\cite{hamner2013muscle}。 \label{fig:9_12}}
\end{figure}


\begin{figure}[!htb]
	\centering
	\includegraphics[width=0.7\linewidth]{chap9/9_13}
	\caption{以 5 米/秒的速度跑步时肌肉力量产生的矢状面关节力矩如图~\ref{fig:9_12}~所示。 \label{fig:9_13}}
\end{figure}


\section{估算关节负荷}

只有已知肌肉力量,才能估算关节负荷。
正如我们在第~\ref{chap:chap8}~章中提到的,区分通过逆动力学分析计算出的净关节反作用力和植入关节的传感器测量的“骨对骨”力至关重要。
只有当我们的肌肉像旋转马达一样施加纯关节扭矩时,我们在第~\ref{chap:chap8}~章中计算出的节段间力才会等于实际关节负荷。
当然,我们的肌肉并不直接产生扭矩,而是产生施加于骨骼的拉力。
这些力在关节中产生力矩以及压缩力和剪切力。
正如我们将看到的,肌肉力量对关节负荷的贡献可能很大。


为了演示肌肉力量如何影响关节负荷,我们考虑在5米/秒的速度下跑步时,垂直地面反作用力达到峰值的瞬间。
为简单起见,我们将使用图~\ref{fig:9_14}~所示的平面模型来计算踝关节的压缩力和剪切力。
在上一节中,我们使用了图~\ref{fig:9_9}~所示的下肢模型来估算腓肠肌和比目鱼肌在此瞬间产生的力。
但请注意,该模型中腓肠肌和比目鱼肌的路径并不位于垂直于踝关节轴线的平面内,因此在接下来的计算中,我们仅保留这些力在图~\ref{fig:9_14}~所示平面上作用的分量。
我们也将地面反作用力投射到这个平面上。
最后,我们假设足部在站立期处于静止状态,并按照第~\ref{chap:chap8}~章中的方法计算力$F_x$和$F_y$:


\begin{figure}[!htb]
	\centering
	\includegraphics[width=0.4\linewidth]{chap9/9_14}
	\caption{平面模型用于估算以 5 米/秒的速度跑步时,垂直\textit{地面反作用力}(ground reaction force, GRF)达到峰值时的踝关节负荷。
		图中所示的力已投射到垂直于踝关节轴的平面上。 \label{fig:9_14}}
\end{figure}

\begin{equation}
	\begin{aligned}
		F_x & = F_x^\text{GAS} + 
				F_x^\text{SOL} + 
				F_x^\text{GRF} \\
			& = 44 \; \text{N} + 495\;\text{N} - 838 \; \text{N} \\
			& = -299 \;\text{N}
	\end{aligned}
	\label{eq:9_3}
\end{equation}


\begin{equation}
	\begin{aligned}
		F_y & = F_y^\text{GAS} + F_y^\text{SOL} + F_y^\text{GRF} \\
		& = 1357 \; \text{N} + 4088 \; \text{N} + 1379 \; \text{N} \\
		& = 6824 \; \text{N}
	\end{aligned}
	\label{eq:9_4}
\end{equation}


请注意,肌肉力量($F^\text{GAS}$ 和 $F^\text{SOL}$)对关节压缩力($F_y$)贡献巨大,约占总压缩力的 80\%。
本例中,跑步时的总压缩力超过 10 个体重,但在动态性较低的活动中也测量到了相当大的压缩负荷。
仪器化的人工关节记录到,行走时髋部峰值接触力约为 2.5 个体重,单腿站立不稳定时髋部峰值接触力超过 5 个体重。
即使在静止的单腿站立姿势下,由于肌肉产生的力量,髋部接触力也超过 2 个体重。
因此,如果我们想要评估关节内的力量,计算肌肉力量至关重要。


\section{动力学优化}

在前面的章节中,我们对待解决的底层优化问题做出了一些假设。
或许最重要的是,我们假设目标函数仅取决于瞬时量。
我们忽略了肌肉-肌腱动力学,并假设肌肉激活(以及由此产生的力量)仅取决于特定时刻所需的净关节力矩。
这类静态优化在计算上相对容易处理,但我们知道人类和动物都会预测未来。
例如,如果某个动作对后续有益,我们可能会在动作开始时投入能量,例如在投球前“收紧”,或在立定跳远前深蹲。
这些爆发性活动充分利用了我们肌肉和肌腱的动力学。
静态优化可能不足以完全理解投掷、跳跃和短跑等动作中的肌肉协调性,因为准确预测某一时间点的肌肉激活可能需要我们考虑整个动作。


我们在前几节中做出的第二个关键假设是,运动学和净关节矩是先验已知的。
例如,如果我们正在研究一个有运动捕捉数据的运动,我们可以使用逆运动学和逆动力学模拟来计算估算关节运动和矩。
然而,合适的实验数据可能无法获得,并且难以(或无法)收集,例如在研究高受伤率的活动或霸王龙的步态时。
此外,我们可能特别希望预测手术、植入物、假肢或外骨骼带来的未观察到的运动适应性,或者发现新的运动策略,以最大限度地提高运动表现或最大限度地减少关节负荷。
我们可以将动态优化用于这些任务。



在动态优化中,我们使用肌肉骨骼动力学模型来确定肌肉协调性及其相应的运动,从而优化运动任务的数学描述。
动态优化通常包括以下步骤:选择候选解,运行正向动态模拟,评估候选解的性能,并不断迭代直至满足某个停止标准(例如,达到目标函数值或已考虑预定数量的候选解;图~\ref{fig:9_15})。
设计变量可能包括肌肉骨骼参数,例如跟腱长度;可穿戴设备的参数,例如运动鞋的刚度;或每块​​肌肉随时间的活动情况。


\begin{figure}[!htb]
	\centering
	\includegraphics[width=0.6\linewidth]{chap9/9_15}
	\caption{动态优化的流程如下:
		选择候选解决方案,运行正向动态模拟,评估候选解决方案的性能,并不断迭代直至满足停止条件。
		候选解决方案可以由任何优化器选择,可以调整模型中的任何参数,并可以根据任何性能指标进行评估。 \label{fig:9_15}}
\end{figure}


例如,假设我们希望优化 100 米短跑过程中的肌肉协调性。
每个候选方案可能描述每块肌肉随时间的活动情况(例如,以参数化曲线或控制点序列的形式)。
我们将力求最大限度地缩短跑完 100 米所需的时间,并可能包含一些防止受伤的约束。
评估目标函数将涉及运行正向动态模拟(参见图~\ref{fig:1_12}),其中模型处于初始配置,在第一个时间点提出的激活应用于模型的肌肉,由此产生的肌肉力应用于骨骼,随后的加速度($\underline{F} = m \underline{a}$)在时间上进行数值积分,以确定模型在下一个时间点的配置,并重复该过程,直到跑步者到达终点线。
解决这个优化问题需要大量的计算资源。
为了使问题更易于处理,我们可以使用仅包含几个关键肌肉的平面模型,专注于固定速度的单个步态周期,或者设计神经控制器并优化相对较少的控制器参数,而不是在每个时间点激活每个肌肉。


目标函数和约束条件可以有多种形式,具体取决于所探讨的研究问题。
在上面的短跑示例中,我们只是试图最小化持续时间。
然而,对于像步行到公交车站或开门这样的任务,更合适的目标函数是将最小化持续时间与最小化总能量消耗结合起来,或许可以用反映活动紧急程度的权重来平衡这些项。
分析其他任务可能需要考虑安全性、舒适性、疲劳度和稳定性等因素的附加项。
需要注意的是,优化器非常“聪明”,它会利用优化问题中任何能够带来更优目标函数值的方面,而不管解决方案的主观性如何——所以要谨慎对待你的期望!我曾经定义过一个目标函数,用于最小化运输成本,而不设置最小行驶距离。
优化器明智地选择直接向前倒下,这消耗的能量很少,但这并不是我想要的。
定义优化问题可能是一个迭代过程,其中优化器返回的解用于识别从研究问题到目标函数和约束条件的转换过程中的错误。


与任何研究一样,验证优化结果至关重要。
如图~\ref{fig:9_6}~所示,计算出的肌肉活动时序可以与实验\textit{肌电图}时序进行比较。
实验运动学、地面反作用力、关节负荷(例如,来自器械关节置换)、间接量热法以及针对相同或类似活动的其他测量方法也可能有用。
动态优化的稳健且经过适当验证的结果可以深入了解运动和肌肉协调性。
例如,B. J. Fregly 利用优化方法发现了一种新的行走方式,可以减轻膝盖的负荷\cite{fregly2007design}。
他学会了以这种步态行走,因此,当他的孩子们去迪士尼乐园时,他能够跟上他们的步伐。


以背越式跳高为例,动态优化可能会发现,跳高运动员在离地时应该抬起双臂,以最大化其重心在起跳时的垂直速度。
同样,在起跳过程中,运动员应该尽可能长时间地与地面接触,以最大化地面反作用力的冲量。
这些洞见是通过背越式跳高首次亮相以来50年来运动员和教练员的经验积累而获得的。
相比之下,在下一节中,我们将描述一个通过计算机建模和数值优化更快地获得此类洞见的示例。



\text{立定跳远过程中的肌肉协调性}

矫形器、假肢和外骨骼等辅助设备有可能通过恢复受伤后的活动能力和提高运动表现来改善人类生活。
然而,我们尚未完全了解辅助设备在我们执行复杂任务时如何与肌肉协调相互作用。
我的学生卡迈克尔$\cdot$翁 (Carmichael Ong) 着手研究模拟辅助设备如何提高立定跳远的表现。
这项任务需要精确的肌肉协调,并且目标明确,因此非常适合进行优化。


Carmichael 使用了一个平面的五段模型(图~\ref{fig:9_16})。
位于踝关节、膝关节、髋关节和肩关节的生理扭矩执行器代表了穿过每个关节的所有肌肉的联合动作,并包含了生物肌肉中依赖于长度和速度的力产生能力。
每个扭矩执行器的活动用一个分段线性函数描述,其节点是动态优化问题中的设计变量。
该问题中的目标函数最大化跳跃距离。Carmichael 使用了一个平面的五段模型(图~\ref{fig:9_16})。
位于踝关节、膝关节、髋关节和肩关节的生理扭矩执行器代表了穿过每个关节的所有肌肉的联合动作,并包含了生物肌肉中依赖于长度和速度的力产生能力。
每个扭矩执行器的活动用一个分段线性函数描述,其节点是动态优化问题中的设计变量。
该问题中的目标函数最大化跳跃距离(通过最小化其负值),同时避免不良结果:
%
\begin{figure}[!htb]
	\centering
	\includegraphics[width=0.8\linewidth]{chap9/9_16}
	\caption{一个包含 5 个段和 7 个自由度的平面模型,用于研究立定跳远过程中的肌肉协调性。
		采用动态优化方法预测踝关节、膝关节、髋关节和肩关节肌肉的协调性,以最大化跳跃距离\cite{ong2015simulation}。 \label{fig:9_16}}
\end{figure}


\begin{equation}
	\begin{bmatrix}
		\text{最小化} & J= & -d & \text{奖励更长的距离} \\
		& & + w_1 (J_\text{fall}) & \text{在着地时惩罚下降} \\
		& & + w_2 (J_\text{injury}) & \text{不鼓励使用韧带} \\
		& & + w_3 (J_\text{slip}) & \text{起跳时惩罚滑行} \\
		& & + w_4 (J_\text{time}) & \text{奖励反向运动}
	\end{bmatrix}
	\label{eq:9_5}
\end{equation}


公式~\ref{fig:9_5}~中的权重 $w_i$ 的选择是为了反映各项的相对重要性,并适当平衡长度、扭矩、力和时间的单位。
确定目标函数中的适当权重通常是一个迭代过程,需要手动调整或使用其他优化器(有时称为“元优化”)。


在本研究中,Carmichael 首先使用\textit{协方差矩阵自适应调整的进化策略}算法在无人协助的情况下优化模型的跳跃距离。
初始优化问题生成了立定跳远模型,该模型捕捉了人类跳跃的显著运动学和动力学特征,包括反向运动、起跳运动学、地面反作用力和关节力矩。
随后,他在臀部、膝盖和脚踝处放置了无质量旋转弹簧,增强了模型的可操作性(图~\ref{fig:9_17})。
当然,这样的装置并不存在,但模拟使我们能够以任何我们想要的方式调整施加的扭矩,这有助于我们了解它们如何影响运动表现。


\begin{figure}[!htb]
	\centering
	\includegraphics[width=1.0\linewidth]{chap9/9_17}
	\caption{立定跳远运动员在无辅助(左)和辅助(右)触地阶段的关节扭矩,其最大限度提升运动表现\cite{horita1991body}。 \label{fig:9_17}}
\end{figure}


在增强场景中,Carmichael 同时优化了每个弹簧的刚度和平衡位置以及其他设计变量。
优化后的辅助装置将跳跃距离增加了1米多,从2.27米增加到3.32米。
令人兴奋的是,该算法能够协同优化装置参数和人体协调性,使两者的综合性能实现了最长的跳跃距离。
这项研究展示了动态优化如何洞察人体运动,并补充辅助装置设计的实验方法。
未来,类似的预测工具或可用于为个体患者定制手术、植入物、假体或辅助装置,或许有助于确定哪种治疗方法能够带来最佳疗效。

\chapter{肌肉驱动模拟} \label{chap:chap10}


预测非常困难,
尤其是关于未来的预测。
\begin{flushright}
	——尼尔斯·玻尔
\end{flushright}

1983年大学毕业后,我的第一份工作是帮助小公司编写计算机辅助设计软件。
当时我在科罗拉多州的一家计算机工厂工作,这家工厂刚刚生产出一台功能强大的新型图形计算机。
在当时,“强大”意味着它每秒可以在小屏幕上画几条线。
如果没有图形软件,没有人会使用我们的图形计算机,所以我的工作就是帮助其他公司的工程师将他们的计算机辅助设计软件在我们新推出的计算机上运行。
我逐渐意识到,几乎所有未来的产品都将在计算机上设计。
在我申请研究生院的时候,我提议开发用于手术设计的计算机图形工具。


两年后,我进入研究生院,有幸加入了斯坦福大学设计组由费利克斯$\cdot$扎亚克领导的生物力学研究实验室。
扎亚克是一位热衷于理解运动控制的神经科学家,他几十年来一直在进行各种精妙的实验,测量跳跃等动作中的肌肉激活模式、地面反作用力以及关节运动。
他意识到,要将这些实验数据整合起来,全面理解运动过程中的肌肉功能,仅仅依靠专业的数据分析是不够的。


仅靠实验测量不足以了解运动过程中的肌肉动作,原因有二。
首先,像肌肉产生的力量这样重要的量通常无法在实验中测量。
其次,仅通过实验观察很难建立因果关系。
例如,可以在行走过程中测量地面反作用力(第~\ref{chap:chap2}~章),并将其用于估计身体重心的加速度。
然而,单靠地面反作用力测量几乎无法了解肌肉如何影响身体重心的加速度,从而无法了解肌肉如何影响行走过程中支撑身体重量和推动身体向前的关键任务。
可以分析 EMG 信号(第~\ref{chap:chap4}~章)来了解肌肉何时活跃,但不能揭示哪些身体运动是由每块肌肉的活动引起的。


我们需要一个新的框架来推进我们对运动过程中肌肉功能的理解。
这个框架必须揭示肌肉激活、肌肉力量、地面反作用力和身体运动之间的关系。


肌肉驱动的运动模拟提供了这一框架。
模拟可以估算肌肉力量,并揭示因果关系,例如行走过程中肌肉对地面反作用力的贡献。
我们还可以利用模拟来预测身体对疾病、手术或肌肉激活改变的反应。
这些能力使我们能够表征运动过程中肌肉的动作,并设计手术和辅助设备。


1985年,当我加入扎亚克的研究小组时,他和他的学生们正处于开发肌肉驱动模拟的前沿。
加入这个小组后,我开启了一段持续30多年的旅程,专注于创建肌肉驱动模拟并进行分析,以改善生物力学受损人群的运动能力。
大学毕业后,我从事的工作在两年内就开发出了用于工程产品的计算机辅助设计工具,而开发用于理解人体运动复杂性的计算机辅助设计工具却成了我毕生的挑战。


本章介绍了我在创建肌肉驱动模拟方面的一些经验。
本章首先阐述了为什么在没有模拟的情况下很难确定运动过程中肌肉的动作,以及为什么文献中充斥着关于肌肉功能的错误结论。
接下来,我们将讨论构建和分析肌肉驱动模拟以正确确定肌肉动作的 4 个阶段。
然后,本章介绍了我和同事开发的开源模拟软件,该软件旨在促进全球合作,让成千上万的研究人员能够构建和共享运动的计算机模拟。
我的目标是通过齐心协力推动这一领域的发展。


\section{理解运动过程中的肌肉动作是一项挑战}

基于肌肉几何形状、肌电图 (EMG) 测量和观察到的运动来推断其动作的实验方法无法正确解释肌肉如何驱动身体。
仅基于解剖学知识的分析常常会导致关于肌肉功能的错误结论。
例如,许多解剖学和生物力学文献将比目鱼肌描述为使踝关节跖屈的肌肉。
比目鱼肌确实会产生踝关节跖屈力矩,从而确实使踝关节跖屈,但该肌肉也能执行其他动作(图 10.1)。
这些动作源于一种称为动态耦合的效应。


动态耦合描述了一个身体节段的运动由于诱导力而影响另一个节段运动的现象。
如图 10.1 所示,比目鱼肌产生的力不仅会产生踝关节跖屈力矩,还会诱导全身节段间的力和关节加速度。
这些节段间的力的大小和方向取决于肌肉施加的力、肌肉的力臂、身体节段的质量和惯性以及身体的姿势。
在图 10.1 右侧的示例中,比目鱼肌产生的力使小腿产生逆时针的角加速度,这需要膝关节向上和向左加速。
大腿及其相邻节段的惯性抵抗了这种加速度,并在膝盖处产生节段间的力,这反过来又加速大腿,依此类推。
因此,尽管比目鱼肌只跨越踝关节,但它却加速了身体的所有关节。


在许多情况下,动态耦合产生的节段间力足够大,从而影响我们对肌肉动作的解读。
虽然远离肌肉的关节处的“肌肉诱导”加速度通常较小,但在附近关节处却可能很大。
例如,Felix Zajac 和 Michael Gordon (1989) 证明,在站立时,比目鱼肌使膝关节伸展的加速度甚至大于使踝关节跖屈的加速度。
此外,他们还指出,双关节肌肉可以诱导与它穿过的其中一个关节产生的力矩相反的关节加速度。
例如,虽然腓肠肌产生膝关节屈曲力矩和踝关节跖屈力矩,但它仍然可以诱导膝关节伸展加速度或踝关节背屈加速度(图 10.2)。
这些看似不协调的加速度在腓肠肌激活时是可能的,因为例如,它产生的膝关节屈曲力矩引起的膝关节屈曲加速度可能会被它产生的踝关节跖屈力矩引起的膝关节伸展加速度所掩盖。
许多生物力学研究在解释肌肉动作时忽略了动态耦合,并得出了错误的结论。
对于由数十个身体节段、关节和肌肉组成的肌肉骨骼系统来说,推断运动过程中肌肉的动作是一项挑战。
需要肌肉驱动的模拟来应对这一挑战。


你可能想知道肌肉是否真的会产生与施加力矩方向相反的加速度。
我以前也曾怀疑过。
史蒂夫$\cdot$皮亚扎是我实验室的一名学生,他创建了肌肉驱动的行走摆动阶段模拟(Piazza and Delp, 1996)。
他的模拟表明,在某些情况下,腘绳肌会产生髋屈曲加速度。
回想一下,腘绳肌在髋后交叉,因此,许多科学家认为这些肌肉总是会产生髋关节伸展。
运动方程分析证实了髋屈曲确实可能产生,但我们的临床同事对此表示怀疑。
尤其是杰奎琳$\cdot$佩里,一位世界领先的肌肉和步态专家,也是我的科学偶像之一,她不相信我们的结果,想要更多证据。
因此,史蒂夫制造了“说服器”,这是一种简单的装置,类似于一条腿,在腘绳肌所在的位置有一根金属丝。
当我们在适当的条件下拉动腘绳肌的金属丝时,髋关节会轻微弯曲。
史蒂夫和我都深信不疑,我们的临床同事,包括佩里博士,也深信不疑。


\section{创建肌肉驱动的模拟}

牛顿运动定律的方程表征了人体的动力学。
我们可以通过求解这些方程来预测人体的运动方式,这个过程被称为动态模拟。
“肌肉驱动”的动态模拟可以预测肌肉产生的力量在行走和跑步等运动过程中如何影响身体各个部位的运动。


开发、测试和分析肌肉驱动模拟的过程包括 4 个阶段(图 10.3)。
在第 1 阶段,您将创建一个计算模型,该模型能够以足够的精度描述肌肉骨骼系统的动态行为,以回答您的研究问题。
如果其他人已经创建并分享了适合您研究的模型,您可以跳过这个繁琐的步骤。
第 2 阶段涉及计算一组肌肉激励,当将这些激励应用于模型时,会生成感兴趣运动的模拟。
第 3 阶段通过将模拟结果与实验测量结果进行比较,确认模拟充分代表了感兴趣的运动。
在第 4 阶段,您将分析模拟以回答您的研究问题。
我们将在接下来的章节中探讨每个阶段。


\section{第一阶段:肌肉骨骼系统动力学建模}

肌肉骨骼动力学模型使我们能够计算由每种肌肉力量引起的运动。
我们四阶段流程的第一阶段是使用描述肌肉激活动力学、肌肉肌腱收缩动力学、肌肉骨骼几何结构和骨骼动力学的方程(图 10.4)来创建肌肉骨骼系统模型。
这些方程表征了肌肉骨骼系统响应肌肉刺激时的时间依赖性行为。


正如我们在第~\ref{chap:chap4}~章中看到的,肌肉兴奋和激活之间的关系受运动单元动作电位和横桥循环的动态控制。
肌肉的激活 ($a$) 可以通过将其时间导数 ($\dot{a}$) 与电流激活和兴奋 ($u$) 关联来建模,如公式 4.1 所示。


肌肉激活是肌肉-肌腱收缩动力学模型的输入,肌肉-肌腱执行器(和)的长度和伸长速度也是如此。
如第~\ref{chap:chap5}~章所述,肌肉和肌腱的动力学受横桥形成的时间过程、肌动蛋白丝的滑动以及肌腱的动力学控制。
肌肉产生的力量 ($F^M$) 并通过其肌腱传递的力量 ($F^T$) 可以用四条无量纲曲线和五个肌肉特定参数来估算,如第~\ref{chap:chap5}~章所述。
当应用于骨骼时,肌肉力量会产生关于关节的力矩,如第~\ref{chap:chap6}~章所述。
肌肉产生的关节力矩导致关节和身体节段加速,从而产生运动。


可以使用身体运动方程来计算身体对肌肉力量和其他负荷的响应加速度:
\begin{equation}
	\underline{\ddot{q}} = M^{-1} (\underline{q})
			\{
				\underline{F}^G (\underline{q}) + 
				\underline{F}^C (q, \dot{q}) + 
				R(\underline{q}) \underline{F}^T (\underline{u}) + 
				\underline{F}^E (\underline{q}, \underline{\dot{q}})
			\}
\end{equation}





\chapter{肌肉驱动步行} \label{chap:chap11}







\chapter{肌肉驱动的跑步} \label{chap:chap12}










\chapter{向前走} \label{chap:chap13}


我们必须有意识地走完一部分通往目标的路程,然后在黑暗中迈向成功。
\begin{flushright}
	——亨利$\cdot$戴维$\cdot$梭罗
\end{flushright}


\begin{figure}[!htb]
	\centering
	\includegraphics[width=1.0\linewidth]{chap13/13_0}
	% 加星号(*)表示不加编号
	\caption*{ \label{fig:13_0}}
\end{figure}

运动生物力学领域取得了长足的进步,但仍有更多领域有待探索。
本书介绍了测量、模型和计算工具如何帮助我们理解人类运动。
我们还展示了生物力学模型如何应用于机器人、跑道设计和外骨骼的设计。
生物力学使我们能够设计保护工人免受伤害的工厂,并制造出关节置换装置,使患有关节疾病的患者能够无痛行走。


这些成就改善了数百万人的生活,但我们也必须意识到现有知识和技术的局限性。
科技的新发展将使我们能够对人们的生活产生进一步的积极影响;
假肢功能的日益增强就是一个很好的例子。
知识的局限性更难克服。
克服这些局限性最重要的是:我们需要创造力,以及正如梭罗所说,勇于探索的勇气。


我想以我的一位合作者取得巨大成功的一次创造性飞跃作为本章的开篇。
接下来,我将描绘一幅我对生物力学领域的未来愿景(这幅愿景当然并不完整)。
你的远见卓识将为这幅充满可能性的画布增添色彩和质感,在那里,你将拥有无限的空间去创造你的杰作。


\section{可穿戴技术}

凯特$\cdot$罗森布鲁斯在我的实验室工作时,在斯坦福医院遇到了一位男士,他因为一种叫做特发性震颤的疾病而无法给妻子写便条,也无法和朋友喝咖啡。
这是一种极其令人沮丧的疾病,但却非常常见。
五六十岁患上特发性震颤的人会因为手抖而无法进行日常生活活动。


这位患有特发性震颤的男子告诉凯特,他尝试过的药物并没有减轻他的手部震颤,而且副作用很大。
他唯一的选择就是手术,在脑部植入刺激电极。
这种疗法被称为深部脑刺激,已被证明对减轻震颤非常有效,但需要进行脑部手术。
脑部手术有严重并发症的风险。
这不是一个可以轻易做出的决定。


凯特知道,他的手颤抖很可能是由大脑特定区域的震荡神经活动引起的,其中包括丘脑腹侧中间核。
凯特找到一些文章,表明电刺激腕部正中神经会引发腹侧中间核内的神经活动。
她推断:刺激腕部的感觉神经或许能激发该脑区活动,从而无需手术即可减轻手部震颤。
作为一项实验,她决定尝试刺激几位自愿接受治疗的特发性震颤患者的正中神经。


这简直是​​一次盲目的尝试。
我们尚不清楚大脑震荡神经活动的确切原因,也不知道腕部电刺激是否会影响它。
但令我们(以及我们的参与者)欣喜的是,刺激正中神经后,他们的手部震颤显著减少。
这些初步数据激励着我们继续前进。
我们继续实验并学习这项新技术。
例如,当以与震颤相同的频率刺激正中神经时,我们发现效果最佳。
凯特与塞雷娜$\cdot$黄(Serena Wong)等人合作,打造了一款可穿戴运动传感器和神经刺激器,其外形类似腕表,可以记录患者手部震颤的频率,并以震颤频率刺激腕部神经。
令人欣慰的是,许多使用该设备的人的手部震颤显著减少\cite{lin2018noninvasive}。
他们恢复了写便条、喝咖啡以及进行其他用握手无法进行的活动的能力(图~\ref{fig:13_1})。


\begin{figure}[!htb]
	\centering
	\includegraphics[width=1.0\linewidth]{chap13/13_1}
	\caption{可穿戴运动传感器和神经刺激器可减少手部震颤。
		Cala Health 的一款可穿戴刺激器(左上)显著减少了手部震颤(下图)。
		一位患者在接受 40 分钟神经刺激治疗后,螺旋画法显著改善(右上)。
		Cala Health, Inc. 是一家斯坦福大学衍生公司,由 Kate Rosenbluth、Serena Wong 和 Scott Delp 共同创立。 \label{fig:13_1}}
\end{figure}


这种新疗法无需药物和手术即可减轻手部震颤。
凯特的创造力和小型惯性测量单元使这项技术成为可能,这些单元可以记录患者的手部运动,使我们能够确定个性化的刺激模式。
神经刺激的效果,以及饮用咖啡或酒精(这些因素会影响手部震颤的程度)和其他因素的影响,可以在患者日常生活中持续数月进行监测。
未来,我们设想成千上万的患者佩戴神经刺激器,记录他们的手部运动并监测治疗效果,从而优化每位患者的功能。


小规模运动传感技术为推进治疗和研究提供了更多机遇。
目前,大多数生物力学实验都是在受控的实验室环境中使用本书所述的技术进行的。
但实验室测量存在严重的局限性。
例如,用于规划脑瘫儿童手术的步态分析通常依赖于在实验室这个陌生环境中仅测量几步的步数,这可能无法代表儿童在日常生活中的能力。
既然我们能够测量实验室外的运动,就可以开始使用在日常生活中几个月内测量的数千步步数来规划治疗方案。
我希望这能帮助我们做出更好的手术决策。


小型传感设备仍有很大改进空间。
目前可穿戴运动监测方法的精度不足以满足许多应用需求,尤其是在涉及运动模式受损人群的应用,准确区分功能性运动和症状性运动至关重要。
现实世界中,长时间佩戴传感器进行生物力学测量会产生大量噪声数据,我们需要新的方法来从这些海量数据集中获取洞见。
这为数据科学和生物力学领域的从业者提供了一个进行有意义互动的机会。


\section{随处可见的物理康复}

走路能力是独立生活的标志。
然而,中风、帕金森病和脑瘫等神经系统疾病和损伤严重限制了人们的活动能力。
骨关节炎等肌肉骨骼疾病会引发疼痛,并限制数百万人的行动能力。
总的来说,行动不便的后果广泛而严重。


康复对于改善行动障碍人士的生活至关重要。
多年来,我在康复诊所工作,深受人们渴望重拾能力的动力以及指导他们康复的治疗师的精湛技艺的鼓舞。
然而,目前的康复方法依赖于诊所内的评估和治疗。
物理治疗师和职业治疗师会评估患者并制定治疗方案,但他们缺乏衡量患者在诊所外活动能力的工具,也缺乏利用数百名类似患者数据来制定治疗方案的方法。
患者需要前往诊所就诊,加上高昂的治疗费用,限制了那些努力康复的患者能够获得的治疗数量。


我们必须开发工具,让患者无需前往诊所即可参与康复。
如果力量训练和活动训练可以在家进行,那么它们的开展频率就会更高。
移动传感器可以测量运动,但提供的信息并不完整。
我们还需要能够应用先进生物力学模型的软件,以精准量化行动障碍患者的运动。
通过优化这些工具,使其能够在智能手机上运行,​​并提供触觉、听觉和视觉反馈,我们可以将智能手机转变为运动测量系统和虚拟物理治疗师。
此外,机器学习方法可以从数千篇科学论文和描述数百万个体的数据(包括视频、临床记录和来自移动传感器的信号)中提取洞见,从而有可能以低成本提供有价值的指导。
对于我们这个领域来说,如何开发这些技术,使其能够为行动障碍患者带来切实的益处,而不是仅仅为了技术而开发技术,这将是一个挑战。


\subsection{大规模实验}

我做过的大多数研究都只有几十名参与者。
几年前,随着全球一些最大的公司开始对生物力学产生兴趣,数百万人开始携带配备运动传感器的智能手机,情况发生了改变。
我曾梦想能够进行大规模的实验。
随着全球大部分人口的手机都配备了加速度计,这个梦想或许可以成为现实。


我和我的合作伙伴与当地一家初创公司 Azumio 合作,收集并分析了来自 46 个国家/地区的 60 多万人的基于智能手机传感器的活动模式,这使得这次调查成为规模最大的体育活动调查,规模大约是前者的 1 千倍。
我们发现,人群中活动最少和最活跃群体之间的不平等可以预测每个人群的肥胖患病率(图~\ref{fig:13_2})。
活动不平等用基尼系数计算,该公式也用于计算收入不平等。
基尼系数的范围从 0(每个人都获得平等的资源份额(完全平等))到 1(一个人获得所有资源)。
在活动不平等严重的国家,通常是因为大量女性相对不活跃;
在这些人群中,女性的预期寿命较低。
我们发现,在步行条件更好的城市,例如餐馆离住所和工作场所都在步行距离内的城市,活动不平等程度较低。
在这些适合步行的城市中,女性的体育活动增幅最大。
这些发现表明,城市规划和公共卫生政策应侧重于增加那些活动量最少的人群的活动量。
我建议生物力学家参与城市和社区规划,将体育锻炼融入日常生活。
这将对数百万人的健康产生巨大的积极影响。


\begin{figure}[!htb]
	\centering
	\includegraphics[width=0.75\linewidth]{chap13/13_2}
	\caption{活动不平等预测肥胖。
		活动不平等程度最高的 5 个国家/地区的个人肥胖可能性比活动不平等程度最低的 5 个国家/地区的个人高出196\% \cite{althoff2017large}。 \label{fig:13_2}}
\end{figure}


预计到 2025 年,全球移动医疗市场规模将增长至 5000 亿美元\cite{dwivedi2025digital}。
智能手机应用程序和可穿戴设备可用于监测各种健康行为,例如体力活动、步行速度、久坐时间、心率和睡眠。
分析可穿戴传感器和应用程序生成的数据,有可能改变我们研究人类行为以及干预改善健康的方式。
这些数据比传统研究收集的数据大得多,而且获取成本低廉。
由于数据是自动收集的,它们可以揭示自然环境中的行为,并惠及通常不参与研究的个体。


诸如体力活动\cite{world2010global}和久坐行为\cite{biswas2015sedentary}等可改变的行为对健康有着显著的影响,但在现代可穿戴传感器出现之前,用于大规模研究这些行为的工具一直有限。
来自应用程序和可穿戴设备的海量数据可以帮助我们发现促使健康行为的环境、社会和个人因素,并找到改善健康的新方法。


移动应用程序、可穿戴设备及其收集的海量数据具有变革性的潜力。
然而,有效分析这些数据需要生物力学、数据科学和公共卫生方面的专业知识。
很少有研究人员在这些领域接受过交叉培训,这使得跨学科的合作与沟通变得困难。
我努力积累数据科学和公共卫生方面的经验,以便能够与这些领域的研究人员交流,并指导那些想要跨越学科界限的学生。
这是一个你可以产生巨大影响的领域,我鼓励你深入研究。


大规模研究需要数据和先进的工具才能从中获得洞见。
遗憾的是,大多数研究实验室不愿共享其数据或工具。
一些公司维护着记录数百万个体活动的数据库,但这些数据库很少与研究人员共享进行分析。
生物力学的未来(乃至数百万个体的健康)要求我们开放这些数据,并开发从这些丰富的数据集中获取洞见的方法。


\section{现代统计学和机器学习}

几乎所有生物医学研究的结论都基于使用参数检验(例如学生 $t$ 检验)的假设检验。
然而,当前数据激增给包括生物力学在内的许多生物医学学科带来了新的挑战。
表征人体运动的数据具有高维性、异构性,并且随着可穿戴传感技术的发展而不断增长。
传统的统计方法限制了我们从这些数据中获得的洞察力。
新的统计方法将改变生物力学,就像它们在自动驾驶、图像识别和自动癌症检测领域所做的那样。


释放大数据集潜能的一种流行方法是机器学习。
Eni Halilaj 搜索了使用机器学习研究运动生物力学的研究论文,发现该领域的出版物数量正在快速增长(图~\ref{fig:13_3})。
随着越来越多的人参与生物力学和机器学习之间的协作和交叉训练,我预计这种趋势将会持续下去。


\begin{figure}[!htb]
	\centering
	\includegraphics[width=0.75\linewidth]{chap13/13_3}
	\caption{近年来,数据科学方法在人体运动生物力学研究中的应用显著增加\cite{halilaj2018machine}。 \label{fig:13_3}}
\end{figure}


Apoorva Rajagopal 及其同事的研究说明了现代统计方法与生物力学建模相结合的影响。
Apoorva 的研究检查了马蹄足步态(用脚尖行走)的手术矫正,马蹄足步态是脑瘫儿童最常见的步态异常之一。
她想知道我们是否可以通过估计步态中的腓肠肌长度来确定哪些肢体的踝关节运动学可能在腓肠肌延长手术后得到改善(图~\ref{fig:13_4})。
她分析了 891 个接受手术的肢体的步态数据,并根据术前和术后踝关节运动学对结果进行分类。
接受腓肠肌延长手术的腓肠肌较短的肢体(“病例”肢体)比腓肠肌不短但仍接受延长手术的肢体(“过度治疗”肢体)更有可能获得良好的手术结果。
差异令人震惊:71\%的病例肢体在后续步态评估中取得了良好的结果,而过度治疗的肢体只有33\%取得了良好的结果(图~\ref{fig:13_5})。
她的研究结果表明:应该使用腓肠肌长度的估计值来判断哪些患者适合手术。

\begin{figure}[!htb]
	\centering
	\includegraphics[width=1.0\linewidth]{chap13/13_4}
	\caption{腓肠肌长度估算方法。
		OpenSim 模型重现了运动捕捉记录的运动(左图),然后计算了腓肠肌内侧肌腱长度(右图)。
		这些长度被标准化为根据典型步态计算的平均峰值长度。
		峰值长度至少比典型平均值低 2 个标准差(即低于右侧面板中阴影带的长度)的肢体被归类为腓肠肌“短”。
		受试者 1(蓝线)的腓肠肌峰值长度较短,而受试者 2(橙线)的腓肠肌峰值长度较短\cite{rajagopal2020pre}。 \label{fig:13_4}}
\end{figure}


\begin{figure}[!htb]
	\centering
	\includegraphics[width=0.7\linewidth]{chap13/13_5}
	\caption{腓肠肌延长手术后的长期疗效。
		病例组肢体接受了符合肌肉骨骼模型的手术,其疗效优于接受过度治疗的肢体。
		观察结果根据术后年数进行分箱,并计算每个分箱的良好疗效率。
		分箱在时间上重叠。
		误差线表示 1 个标准差;圆圈面积与观察次数成正比\cite{rajagopal2020pre}。 \label{fig:13_5}}
\end{figure}


\section{建立神经肌肉控制模型来预测运动}

模拟大脑如何协调运动是生物力学领域的一大挑战。
计算建模大多局限于重现观察到的运动,以研究无法测量的量,例如肌肉力量和关节负荷。
现代计算技术推动了预测模拟的发展,这种模拟无需事先收集实验数据即可生成运动模拟。
预测模拟使用优化来实现高级目标,例如最小化模型在步态过程中的能量消耗,同时遵循牛顿定律等物理约束和肌肉激活延迟等生理约束。
优化问题中的变量描述了如何协调模型的肌肉。
只要优化问题能够捕捉到真实运动的显著特征,解决方案就能生成一个模拟,该模拟代表自然界中观察到的运动。
预测模拟将使我们能够设计出更好的假肢和手术,从而最大限度地发挥个体的功能。


为了生成预测模拟,我们必须解决高维优化问题。
解决这些问题的一种策略是使用强化学习,它无需了解底层模型,也无需设计控制器的结构:优化器只需学习如何生成模型输入(在本例中是随时间变化的肌肉刺激),从而产生理想的输出。
Łukasz Kidziński\cite{learning2018learning}在过去两届神经信息处理系统大会期间举办了一场精彩的竞赛,超过 1 千名参赛者使用强化学习来解决运动控制中的经典问题。
例如,参赛者面临的挑战是开发控制器,使肌肉骨骼模型能够在不平坦的地形上奔跑。
结果令人惊叹!
参赛者证明了强化学习技术可以合成走路和跑步等自然动作。
他们的优化器还发现了各种奇特的跳跃和跳跃步态。
该项目的成功得益于生物力学家和计算机科学家的合作,以及 OpenSim 等强大的开源软件和 CrowdAI 的强化学习环境。


直接配点法是解决生物力学中高维优化问题的另一种有效方法。
直接配点法将运动方程转换为代数约束方程,有效地同时计算所有时间帧。
解决优化问题极具挑战性,研究人员通常需要等待数天甚至数周才能获得优化结果。
直接配点法已被用于解决生物力学中的最优控制问题,其速度远快于传统方法,并将使预测模拟的用途更加广泛。
我确信直接配点法将对生物力学的未来产生巨大影响,并鼓励您进一步了解它。


\section{激励运动}

我们都知道,体育锻炼有益健康。
正如希波克拉底的名言:“走路是人类最好的良药。” 
然而,近一半的成年人未能获得足够的运动来维持健康,这种缺乏运动的状况正在给我们带来高昂的代价。
全球每年有 500 万人死于心脏病、中风和糖尿病等疾病,而其中许多疾病只需增加体育锻炼即可预防\cite{lee2017reducing}。
能够激励健康行为的技术(尤其是像运动这样已被广泛接受且成本较低的行为),可以改善数百万人的健康,并减轻缺乏体育锻炼带来的经济负担。


智能手机和可穿戴传感器有潜力实现这一愿景,并彻底改变医疗保健。
我之前提到过,这些设备提供了大量关于身体活动的数据,智能手机能够与用户进行近乎持续的互动,从而激励健康行为。
然而,智能手机和其他设备并没有改善健康状况,有些甚至产生了相反的效果\cite{jakicic2016effect}。
显然,单靠技术能力不足以解决这个问题。
如果我们想让这些设备成为改善公共健康的积极工具,而不仅仅是收集信息,我们就必须了解人们如何与科技互动。
激励个人远比仅仅开发一个漂亮的新设备要复杂得多。


身体活动是一种强效且低成本的良药,但我们需要精准且引人入胜的工具来激励它。
行为心理学、人机交互理论和生物力学等不同学科的理念已在小范围内取得成功;
然而,这些创新很少被融入移动健康应用中。
生物力学家必须与其他学科的专家合作,探索激励身体活动的新方法,以惠及全球数百万人,并解决神经系统残疾等亟需解决的领域。
我们的愿景是实现有效且价格合理的医疗保健,并在全球范围内推广。


\section{开放科学}

开放科学是推动我们领域发展的最佳途径。
作为一个群体,我们必须向所有人免费开放出版物、数据、模型和软件。
我们已经从健康个体以及患有各种运动障碍的个体中收集了海量高保真运动数据。
但这些数据隐藏在计算机硬盘中,无法供整个群体访问。
匿名化并发布这些数据将使前所未有的生物力学研究成为可能。
为了鼓励数据共享和开放科学,我和我的同事创建了一个用于共享生物力学数据、模型和软件的网站——\href{simtk.org}{simtk.org},该网站托管着超过 1000 个共享资源的项目。
本书中介绍的模型和数据均可在此网站上获取。
生物力学文化必须持续变革,使数据共享成为常态。


共享软件也能加速研究进程。
生物力学模拟正迅速普及,部分原因在于它是实验方法的有力补充,同时也因为现在已经有了成熟的软件工具。
正如我们在本书中所见,OpenSim 软件使用户能够构建生物力学模型、模拟肌肉骨骼动力学、预测新动作并传播新的计算工具。
Jeff Rankin、Jonas Rubenson 和 John Hutchinson(2016)使用 OpenSim 研究了鸵鸟(奔跑速度最快的双足动物)拥有惊人速度、敏捷性和效率的机制(图~\ref{fig:13_6})。
Taymaz Homayouni 及其同事\cite{homayouni2015modeling}使用 OpenSim 制作了新的可植入机制原型,以恢复手臂部分瘫痪患者的功能。
Matt DeMers、Jennifer Hicks 和我使用 OpenSim 比较了反射和准备性肌肉协同激活在预防踝关节损伤方面的有效性(图~\ref{fig:13_7}),这项研究如果通过实验进行会过于危险。
这些只是 OpenSim 软件支持的数百项研究中的一小部分,我很高兴看到 OpenSim 社区不断发展壮大。


\begin{figure}[!htb]
	\centering
	\includegraphics[width=0.9\linewidth]{chap13/13_6}
	\caption{鸵鸟的 OpenSim 模型,鸵鸟是地球上跑得最快的双足动物\cite{rankin2016inferring}。 \label{fig:13_6}}
\end{figure}


\begin{figure}[!htb]
	\centering
	\includegraphics[width=1.0\linewidth]{chap13/13_7}
	\caption{用于研究踝关节损伤的斜坡落地模拟帧。OpenSim 模型来自 DeMers 等人\cite{demers2017preparatory}。 \label{fig:13_7}}
\end{figure}


除了开源软件之外,肌肉骨骼结构计算模型的开放获取也促进了广泛的研究。
例如,Sam Hamner 的跑步模拟模型(图~\ref{fig:12_4})、Miguel Christophy 的腰椎模型\cite{christophy2012musculoskeletal}以及 Kate Saul 的上肢模型(图~\ref{fig:10_6})已被成千上万的研究人员和学生使用。
持续推进开放科学的趋势将有助于研究日益复杂的问题,这些问题只有通过拥有不同专业知识的人员的共同努力才能解决。
OpenSim 项目提供了一个协作研究平台,支持一个多元化、全球化且不断扩张的社区,致力于解决生物力学领域最重要的问题(图~\ref{fig:13_8})。
我希望您能加入这个团队。


\begin{figure}[!htb]
	\centering
	\includegraphics[width=1.0\linewidth]{chap13/13_8}
	\caption{最近一年内访问 OpenSim 文档的访客所在地。
		自 2012 年上线以来,OpenSim 维基文档每年吸引来自世界各地的超过 25,000 名用户访问\cite{seth2018opensim}。 \label{fig:13_8}}
\end{figure}


\section{接过接力棒}

我希望本书能加深您对人体运动的优雅和复杂性的理解。
即使我们涵盖的有限内容,也是数千人数百年来共同研究的结晶。
这是一个令人惊叹的团队合作。
正如沃尔特$\cdot$惠特曼所写:“精彩的演出仍在继续,而您或许可以贡献一首诗。” 
的确,您通过本书所学到的原理,为您参与该领域奠定了基础。
您可以用全新的视角阅读有关跑鞋和机器人的新闻文章和期刊论文。
您可以参加研讨会或会议,或者只是和朋友聊聊我们的运动方式。
只要稍加创意,再加一点努力,您就可以通过生物力学为世界做出有意义的贡献。
让我们继续前进!











\label{chap:preface}
\begin{table}[htbp]
	\newcommand{\tabincell}[2]{\begin{tabular}{@{}#1@{}}#2\end{tabular}} %换行指令
	\centering
	\caption{名词列表 \label{tab:0_1}}
	\renewcommand\arraystretch{1.0}	%设置表格内行间距
	\setlength{\tabcolsep}{8mm}{
	\begin{tabular}{llll}
		\toprule 
		 名词(缩略词)   && 定义 \\
		 
		 \midrule
		 Abbott && 艾博特 \\
		 
		 \midrule
		 adenosine diphosphate (ADP) && 二磷酸腺苷 \\
		 
		 \midrule
		 adenosine triphosphate (ATP) && 三磷酸腺苷 \\
		 
		 \midrule
		 cost of transport && 单位距离能耗 \\
		 
		 \midrule
		 cross-bridge cycle && 横桥循环 \\
		 
		 \midrule
		 dorsiflexion && 背屈 \\
		 
		 \midrule
		 extensor digitorum longus && 伸趾长肌 \\
		 
		 \midrule
		 foot progression angle && 足前进角 \\
		 
		 \midrule
		 gastrocnemius && \href{https://baike.baidu.com/item/%E8%85%93%E8%82%A0%E8%82%8C}{腓肠肌} \\
		 
		 \midrule
		 gluteus maximus && 臀大肌 \\
		 
		 \midrule
		 hamstrings && \href{https://baike.baidu.com/item/hamstring/51109946}{腘绳肌腱} \\
		 
		 \midrule
		 Kat Steele && 凯特$\cdot$斯蒂尔 \\
		 
		 \midrule
		 Matthew Abbate && 马修$\cdot$阿贝特 \\
		 
		 \midrule
		 Metabolic power && 代谢功率 \\
		 
		 \midrule
		 Molly Seamans && 莫莉$\cdot$西曼斯 \\
		 
		 \midrule
		 pentapedal gait && 五足步态 \\
		 
		 \midrule
		 % zhi2
		 plantarflexion && \href{https://baike.baidu.com/item/%E8%84%9A%E5%BA%95%E5%BC%AF%E6%9B%B2}{跖屈} \\
		 
		 \midrule
		 power stroke && 动力冲程 \\
		 
		 \midrule
		 rectus femoris   && \href{https://baike.baidu.com/item/%E8%82%A1%E7%9B%B4%E8%82%8C}{股直肌} \\
		 
		 \midrule
		 rigor state   && 强直结合状态 \\
		 
		 \midrule
		 sarcoplasmic reticulum   && \href{https://baike.baidu.com/item/%E8%82%8C%E8%B4%A8%E7%BD%91/7911541}{肌质网} \\
		 
		 \midrule
		 Silvia Blemker   && 西尔维亚$\cdot$布莱姆克尔 \\
		 
		 \midrule
		 soleus   && \href{https://baike.baidu.com/item/%E6%AF%94%E7%9B%AE%E9%B1%BC%E8%82%8C}{比目鱼肌} \\
		 
		 \midrule
		 stance phase   && 支撑阶段 \\
		 
		 \midrule
		 step length   && 步长 \\
		 
		 \midrule
		 step width   && 踏步宽度 \\
		 
		 \midrule
		 stride length   && 步幅 \\
		 
		 \midrule
		 swing phase && 摆动阶段 \\
		 
		 \midrule
		 terminal cisternae && 终末池 \\
		 
		 \midrule
		 tibialis anterior && \href{https://baike.baidu.com/item/%E8%83%AB%E9%AA%A8%E5%89%8D%E8%82%8C}{胫骨前肌} \\
		 
		 \midrule
		 tibialis posterior && 胫骨后肌 \\
		 
		 \midrule
		 toe-out angle && 足偏脚 \\
		 
		 \midrule
		 total knee replacement && \href{https://baike.baidu.com/item/%E5%85%A8%E8%86%9D%E5%85%B3%E8%8A%82%E7%BD%AE%E6%8D%A2%E6%9C%AF/15634686}{全膝关节置换术} \\
		 
		 \midrule
		 transverse tubule (T-tubule) && \href{https://baike.baidu.com/item/%E6%A8%AA%E6%96%AD%E7%AE%A1/10802377}{横断管} \\
		 
		 \midrule
		 unfused tetanus && 不完全强直收缩 \\
		 
		 % 股肌(Vasti)(除股直肌以外的其他三条股四头肌)起于股骨体,而臀大肌则止于股骨体的后侧面
		 \midrule
		 Vasti && 股肌 \\
		 
		 \midrule
		 Z-line && \href{https://baike.baidu.com/item/Z%E7%BA%BF/564097}{Z线} \\

		\bottomrule  

	\end{tabular}}
\end{table}%





\begin{table}[htbp]
	\newcommand{\tabincell}[2]{\begin{tabular}{@{}#1@{}}#2\end{tabular}} %换行指令
	\centering
	\caption{生物学和生理学的术语表 \label{tab:0_2}}
	\renewcommand\arraystretch{1.0}	%设置表格内行间距
	\setlength{\tabcolsep}{8mm}{
		\begin{tabular}{ll}
			\toprule 
			术语   & 用法和同义词 \\
			\midrule
			先进   & 与祖先状况不同  \\
			\midrule
			类比     & 具有共同功能的两个或多个物种的结构或行为   \\
			\midrule
			类人猿     & \makecell{一组灵长类动物,包括所有现代猴子、猿和人类,\\以及祖先类人猿的已灭绝后代}    \\
			\midrule
			注意力      &对可用信息、感官或助记符的子集的增强处理   \\
			\midrule
			狭鼻类动物       & 一组灵长类动物,包括旧世界的猴子、猿和人类   \\
			\midrule
			选择       &在备选方案中选择目标或行动   \\
			\midrule
			结合       & 代表性元素的组合   \\
			\midrule
			当前上下文       &感官输入和最近发生的事件   \\
			\midrule
			决策      &对世界的看法   \\
			\midrule
			情景记忆       &回忆事件,暗示意识   \\
			\midrule
			事件      &特定时间和地点的情境、目标、行动和结果的一次性结合   \\
			\midrule
			外部指导       &基于外部感官输入的行为  \\
			\midrule
			目标       &作为动作目标的物体或地方   \\
			\midrule
			习惯      &过度训练的结果,在不参考预测结果的情况下对刺激产生响应   \\
			\midrule
			类人猿亚目      &一组灵长类动物,包括眼镜猴和类人猿   \\
			\midrule
			同源性      &由于共同祖先的遗传而出现在两个或多个物种中的特征   \\
			\midrule
			“内部”指导      &当没有感官输入提示行为时   \\
			\midrule
			记忆       &存储信息   \\
			\midrule
			需要      &食物和液体等生物学要求; 同义词:动力、动机   \\
			\midrule
			结果      &刺激或行为产生的好处或伤害   \\
			\midrule
			优势响应、行为      &天生的、习惯性的或条件反射   \\
			\midrule
			原始     &类似于祖先的情况   \\
			\midrule
			前瞻、前瞻记忆、前瞻编码       &短期记忆中目标的表示   \\
			\midrule
			强化       &作为反馈的结果   \\
			\midrule
			再表示       &基于其他低阶表示的神经表示   \\
			\midrule
			响应       &依赖于与刺激或结果的条件关联的行动   \\
			\midrule
			奖励       &有益的结果   \\
			\midrule
			规则      &行为输入输出算法   \\
			\midrule
			符号      &小于整个对象但大于基本感官特征的非空间提示   \\
			\midrule
			策略      &(1) 一个问题的两个或多个解决方案中的一个; (2) 部分解决问题   \\
			\midrule
			值      &成本或收益的程度   \\
			\bottomrule  
			
	\end{tabular}}
\end{table}%




\nocite{*} 
\printbibliography


\end{document}
